%%
%% This is file `pstricks-add.tex',
%%
%% IMPORTANT NOTICE:
%%
%% Package `pstricks-add.tex'
%%
%% Herbert Voss <voss _at_ pstricks.de>
%%
%% This program can be redistributed and/or modified under the terms
%% of the LaTeX Project Public License Distributed from CTAN archives
%% in directory macros/latex/base/lppl.txt.
%%
%% DESCRIPTION:
%%   `pstricks-add' is a PSTricks package for additionals to the standard
%%         pstricks package
%%
\csname PSTricksAddLoaded\endcsname
\let\PSTricksAddLoaded\endinput
%
% Requires some packages
\ifx\PSTricksLoaded\endinput\else\input pstricks \fi
\ifx\PSTplotLoaded\endinput\else\input pst-plot \fi
\ifx\PSTnodesLoaded\endinput\else\input pst-node \fi
\ifx\MultidoLoaded\endinput\else\input multido \fi
\ifx\PSTXKeyLoaded\endinput\else\input pst-xkey \fi
%
\def\fileversion{2.28}
\def\filedate{2004/11/24}
\message{`pstricks-add' v\fileversion, \filedate\space (hv)}
%
\edef\PstAtCode{\the\catcode`\@} \catcode`\@=11\relax
\SpecialCoor
\pst@addfams{pstricks-add}
%
%% prologue for postcript
\pstheader{pst-eqdf.pro}%
%
%
%-------------------------------- pstricks ------------------------------------
%
%%%%%%% \begin{pspicture} %%%%%%%%%%%%%%%%%%%
\newdimen\pst@dimm%
\newdimen\pst@dimn%
\newdimen\pst@dimo%
\newdimen\pst@dimp%
\newcount\pst@cntm%
\newcount\pst@cntn%
\newcount\pst@cnto%
\newcount\pst@cntp%
\psset{dx=0}%
\psset{dy=0}%
\newcount\@zero\@zero=0\relax
%

\def\pst@getdimdim#1 #2 #3\@nil{%
  \pssetlength\pst@dimm{#1}%
  \ifx#2\pst@missing\relax%
    \pst@dimn=\pst@dimm\pst@dimm=-\pst@dimn%
  \else%
    \pssetlength\pst@dimn{#2}%
  \fi%
}
\def\pst@getxdimdim#1 #2 #3\@nil{%
  \pssetxlength\pst@dimm{#1}%
  \ifx#2\pst@missing\relax%
    \pst@dimn=\pst@dimm\pst@dimm=-\pst@dimn%
  \else%
    \pssetxlength\pst@dimn{#2}%
  \fi%
}
\def\pst@getydimdim#1 #2 #3\@nil{%
  \pssetylength\pst@dimm{#1}%
  \ifx#2\pst@missing\relax%
    \pst@dimn=\pst@dimm\pst@dimm=-\pst@dimn%
  \else%
    \pssetylength\pst@dimn{#2}%
  \fi%
}
%
% A modulo macro for integer values
% \pst@mod{34}{6}\value ==> \value is 4
%
\def\pst@mod#1#2#3{%
  \begingroup%
  \pst@cntm=#1\pst@cntn=#2\relax%
  \pst@cnto=\pst@cntm%
  \divide\pst@cntm by \pst@cntn%
  \multiply\pst@cntn by \pst@cntm%
  \advance\pst@cnto by -\pst@cntn%
  \edef\value{\endgroup\def\noexpand#3{\number\pst@cnto}}\value%
}
\def\pst@max#1#2#3{%
  \begingroup%
  \pst@cntm=#1\pst@cntn=#2\relax%
  \ifnum\pst@cntm<\pst@cntn\pst@cntm=\pst@cntn\fi
  \global#3=\the\pst@cntm%
  \endgroup%
}
\def\pst@maxdim#1#2#3{%
  \begingroup%
  \pst@dimm=#1\pst@dimn=#2\relax%
  \ifdim\pst@dimm<\pst@dimn\pst@dimm=\pst@dimn\fi
  \global#3=\the\pst@dimm%
  \endgroup%
}
\def\pst@abs#1#2{%
  \begingroup%
  \pst@cntm=#1\relax%
  \ifnum\pst@cntm<\z@\pst@cntm=-\pst@cntm\fi%
  \global#2=\the\pst@cntm
  \endgroup%
}
\def\pst@absdim#1#2{%
  \begingroup%
  \pst@dimm=#1\relax%
  \ifdim\pst@dimm<\z@\pst@dimm=-\pst@dimm\fi%
  \global#2=\the\pst@dimm%
  \endgroup%
}
%
%--------------------------------------- small stuff -------------------------------
\def\defineTColor{\@ifnextchar[{\defineTColor@i}{\defineTColor@i[]}}
\def\defineTColor@i[#1]#2#3{%     transparency "Colors"
  \newpsstyle{#2}{%
     fillstyle=vlines,hatchcolor=#3,
     hatchwidth=0.1\pslinewidth,hatchsep=1\pslinewidth,#1%
  }%
}
%
\def\rmultiput{\def\pst@par{}\pst@ifstar{\@ifnextchar[{\rmultiput@i}{\rmultiput@i[]}}}
\def\rmultiput@i[#1]{\begingroup\psset{#1}\rmultiput@ii}
\def\rmultiput@ii#1{\def\@rmultiputArg{#1}%
  \@ifnextchar({\rmultiput@iii}{\rmultiput@iii(\z@,\z@)}}
\def\rmultiput@iii(#1){%
  \pst@killglue%
  \if@star\rput*(#1){\@rmultiputArg}
  \else\rput(#1){\@rmultiputArg}%
  \fi
  \@ifnextchar({\rmultiput@iii}{\endgroup}%
}
%%%%%%%%%%%%%%%%%%%%%%%%%%%%%%%%%%%%%%%%%%%%%%%%%%%%%%%%%%%%%%%%%%%%%%
%% Begin patch TN; Date (YY-MM-DD): 04-07-17; first part
\define@key[psset]{}{dash}{%	defined in pstricks.tex
  \pst@expandafter\psset@@dash{#1}\@nil% Error handling for empty argument.
}
\define@key[psset]{pstricks-add}{maxdashes}{\edef\psk@maxdashes{#1}}
\psset{maxdashes=11}
\def\psset@@dash#1\@nil{%
 \def\psk@dash{}%
 % modification 04-08-07:
 \pst@cntm0
 \def\next##1 ##2\relax{%
   \expandafter\ifnum\psk@maxdashes>\pst@cntm\relax% 04-08-07
   \edef\@tempa{##1}%
   \ifx\@tempa\@empty\else% gobble leading spaces
   \pssetlength\pst@dimc{##1}%
       \advance\pst@cntm by 1
   \edef\psk@dash{\psk@dash\space\pst@number\pst@dimc}%
   \fi%
   \edef\@tempa{##2}%
   \ifx\@tempa\@empty\else% detect end
   \ifx\@tempa\space\else% gobble trailing spaces
     \next##2\relax%
   \fi\fi%
   \else% 04-08-07
     \@pstrickserr{Number of dashes > \psk@maxdashes. Increasing
`maxdashes' might work.}\@ehpa% 04-08-07
   \fi% 04-08-07
 }%
\expandafter\next#1 \relax}
%% End patch TN; Date (YY-MM-DD): 04-07-17; 1st part
%%%%%%%%%%%%%%%%%%%%%%%%%%%%%%%%%%%%%%%%%%%%%%%%%%%%%%%%%%%%%%%%%%%%%%
%
\psset{dash=5pt 3pt}% black white black white 
%%%%%%%%%%%%%%%%%%%%%%%%%%%%%%%%%%%%%%%%%%%%%%%%%%%%%%%%%%%%%%%%%%%%%%
%% Begin patch TN; Date (YY-MM-DD): 04-07-17; 2nd part
\def\psls@dashed{%
  \ifpsdashadjust
    [ \psk@dash ] \pst@linetype\space \tx@DashLine
  \else
    [ \psk@dash ] 0 setdash stroke
  \fi}
%% End patch TN; Date (YY-MM-DD): 04-07-17; 2nd part
%%%%%%%%%%%%%%%%%%%%%%%%%%%%%%%%%%%%%%%%%%%%%%%%%%%%%%%%%%%%%%%%%%%%%%
%-----------------------------------------------------------------------------------
\pst@def{DashLine}<{
    dup 0 gt
    { /a .5 def PathLength exch div }
    { pop /a 1 def PathLength } ifelse
    /b ED % pattern should fit evenly in b
    dup /X ED % pattern array
    0 get /y ED % length of first black segment
    /z 0 X {add} forall def % length of the full pattern
    %% Computation of the scaling factor as described by van Zandt:
    b a .5 sub 2 mul y mul sub z Div round
    z mul a .5 sub 2 mul y mul add b exch Div
    %%%% scaling factor on stack.
    /z ED %% now, z is the scaling factor
    false % for the length test below
    X { z mul } forall X astore %% modification TN 04-08-07
    %%% Checking whether at least one dash in X has positive length:
    {0 gt or} forall
    { X 1 a sub y mul }
    { [ 1 0 ] 0 }
    ifelse
    setdash stroke }>
%% TN end patch; Date (YY-MM-DD): 04-07-17;
%%%%%%%%%%%%%%%%%%%%%%%%%%%%%%%%%%%%%%%%%%%%%%%%%%%%%%%%%%%%%%%%%%%%%%
%
%
\define@key[psset]{pstricks-add}{intSeparator}{\edef\psk@intSeparator{#1}}
\psset{intSeparator={,}}
%
\def\psFormatInt{\@ifnextchar[{\psFormatInt@i}{\psFormatInt@i[]}}
\def\psFormatInt@i[#1]#2{{%
    \def\ps@tempa{#1}
    \ifx\ps@tempa\@empty\else\psset{#1}\fi
    \count1=#2\count2=\count1
    \ifnum\count1=0 0\else
	\ifnum\count1>999999
	    \count3=\count1
	    \divide\count3 by 1000000
	    \the\count3\psk@intSeparator
	    \multiply\count3 by 1000000
	    \advance\count1 by -\count3 % modulo 1000000
	\fi
	\ifnum\count2>999
	    \count3=\count1
	    \divide\count3 by 1000
	    \ifnum\count2>99999
		\ifnum\count3<100 0\fi
		\ifnum\count3<10 0\fi
	    \fi
	    \the\count3\psk@intSeparator
	    \multiply\count3 by 1000
	    \advance\count1 by -\count3 %modulo 1000
	\fi%
	\ifnum\count2>999
	    \ifnum\count1<100 0\fi
	    \ifnum\count1<10 0\fi
	\fi
	\the\count1
    \fi
}}
%
\newdimen\pst@braceWidth
\define@key[psset]{pstricks-add}{braceWidth}{\pssetlength\pst@braceWidth{#1}}
\define@key[psset]{pstricks-add}{bracePos}{\edef\psk@bracePos{#1}}
\psset{braceWidth=10pt,bracePos=0.5}
%
\def\@@rput@iv(#1){\pst@killglue\pst@makebox{\@@rput@v{#1}}}
\def\@@rput@v#1{%
  \begingroup
    \use@par
    \pst@makesmall\pst@hbox
    \pst@Verb{%
      Alpha 90 sub \pst@number\pst@braceWidth  0 lt {180 add} if
      \ifx\psk@rot\@empty\else\psk@rot add \fi
      /rotAngle exch def
    }%
    \setbox\pst@hbox=\hbox{%
      \pst@Verb{rotAngle \tx@RotBegin}%
      \box\pst@hbox\pst@Verb{\tx@RotEnd}%
    }%
    \psput@{#1}\pst@hbox
  \endgroup
  \ignorespaces%
}
%
\def\psbrace{\pst@object{psbrace}}
\def\psbrace@i(#1)(#2)#3{{%
  \psset{linearc=2mm,linewidth=1pt,%
    nodesepA=0pt,nodesepB=0pt,bracePos=0.5}% the default
  \use@par%
  \pst@getcoor{#1}\pst@tempa%
  \pst@getcoor{#2}\pst@tempb%
  \pnode(!%
    /bW2 \pst@number\pst@braceWidth \pst@number\psyunit div 2.0 div def
    \pst@tempa /YA exch \pst@number\psyunit div def
    /XA exch \pst@number\psxunit div def
    \pst@tempb /YB exch \pst@number\psyunit div def
    /XB exch \pst@number\psxunit div def
    /Alpha YB YA sub XB XA sub atan def
    /xMid XB XA sub \psk@bracePos\space mul XA add def
    /yMid YB YA sub \psk@bracePos\space mul YA add def
    /@deltaX Alpha sin bW2 mul def
    /@deltaY Alpha cos bW2 mul def
    /@xTemp xMid @deltaX 2 mul add def
    /@yTemp yMid @deltaY 2 mul sub def
    @xTemp @yTemp){@tempNode}%
  \pst@getcoor{@tempNode}\pst@tempc%
  \@@rput@iv(! %
    \pst@tempc /Yc exch def
    /Xc exch def
    Xc \psk@nodesepA\space add \pst@number\psxunit div
    Yc \psk@nodesepB\space add \pst@number\psxunit div ){#3}
  \psline(#1)%
    (! XA @deltaX add YA @deltaY sub)%
    (! @xTemp @deltaX sub @yTemp @deltaY add)(@tempNode)%
  \psline(@tempNode)%
    (! @xTemp @deltaX sub @yTemp @deltaY add)%
    (! XB @deltaX add YB @deltaY sub)(#2)%
}\ignorespaces}
%
% -------------- the arrow part -------------
%  the original table
% \def\pst@arrowtable{,<->,<<->>,>-<,>>-<<,(-),[-]}
\edef\pst@arrowtable{\pst@arrowtable,)-(,]-[,|>-<|,H-H}
%
% HookLeft/RightArrow
\newdimen\pshooklength
\newdimen\pshookwidth
\define@key[psset]{pstricks-add}{hooklength}{\pssetlength\pshooklength{#1}}
\define@key[psset]{pstricks-add}{hookwidth}{\pssetlength\pshookwidth{#1}}
\psset{hooklength=3mm,hookwidth=1mm}
%
\edef\pst@arrowtable{\pst@arrowtable,H-H} % add new arrow
\def\tx@RHook{RHook }         % PostScript name
\@namedef{psas@H}{%
  /RHook {
    /x ED                     % hook width
    /y ED                     % hook length 
    /z CLW 2 div def          % save it
    x y moveto                % goto first point
    x 0 0 0 0 y 
    curveto                   % draw Bezier
    stroke 
    0 y moveto                % define current point
  } def
  \pst@number\pshooklength
  \pst@number\pshookwidth
  \tx@RHook 
}
\@namedef{psas@<|}{%
    \psk@tbarsize\space \tx@Tbar
    0 CLW 2 div T
    newpath
    true \psk@arrowinset\space \psk@arrowlength\space \psk@arrowsize\space \tx@Arrow%
}
% ]-[ arrow
\def\tx@BracketOut{BracketOut }
\@namedef{psas@[}{%
  /BracketOut {%
  CLW mul add dup CLW sub 2 div
%/x ED mul CLW add
  /x ED mul neg
  /y ED
  /z CLW 2 div def
  x neg y moveto
  x neg CLW 2 div L x CLW 2 div L x y L stroke 0 CLW moveto } def
  \psk@bracketlength\space \psk@tbarsize\space \tx@BracketOut
}
% )-( arrow
\def\tx@RoundBracketOut{RoundBracketOut }
\@namedef{psas@(}{%
  /RoundBracketOut {%
    CLW mul add dup 2 div
%/x ED mul
    /x ED mul neg
    /y ED
    /mtrx CM def
    0 CLW
    2 div T x y mul 0 ne { x y scale } if
    1 1 moveto
    .85 .5 .35 0 0 0 curveto
    -.35 0 -.85 .5 -1 1 curveto
    mtrx setmatrix stroke 0 CLW moveto } def
  \psk@rbracketlength\space \psk@tbarsize\space \tx@RoundBracketOut
}
%
% Redefininition of \psset@arrowscale to store value of X scale factor
%\define@key[psset]{pstricks-add}{arrowscale}{\pst@getscale{#1}\psk@arrowscale}
%\psset{arrowscale=1}

\define@key[psset]{}{arrowscale}{%
  \pst@@arrowscale@i#1 \@nil
  \pst@getscale{\pst@arrowscale}\psk@arrowscale%
}
\def\pst@@arrowscale@i#1 #2\@nil{\edef\pst@arrowscale{#1}}
\psset{arrowscale=1}

% New parameter "arrowfill", with default as "true"
\newif\ifpsArrowFill
\define@key[psset]{pstricks-add}{ArrowFill}{\@nameuse{psArrowFill#1}}
\psset{ArrowFill=true}

% Modification of the PostScript macro Arrow to choose to fill or not the arrow
% (it require to restore the current linewidth, despite of the scaling)
\pst@def{Arrow}<{%
    CLW mul add dup 2 div
    /w ED mul dup
    /h ED mul
    /a ED { 0 h T 1 -1 scale } if
    gsave
    \ifpsArrowFill\else\pst@number\pslinewidth \pst@arrowscale\space div SLW \fi
    w neg h moveto
    0 0 L w h L w neg a neg rlineto
    \ifpsArrowFill gsave fill grestore \else gsave closepath stroke grestore \fi
    grestore
    0 h a sub moveto
}>
%
\define@key[psset]{pstricks-add}{nArrowsA}{\def\psk@nArrowsA{#1}}
\define@key[psset]{pstricks-add}{nArrowsB}{\def\psk@nArrowsB{#1}}
\define@key[psset]{pstricks-add}{nArrows}{\def\psk@nArrowsA{#1}\def\psk@nArrowsB{#1}}
\psset{nArrows=2}
%
\@namedef{psas@>>}{%
    \psk@nArrowsA\space 1 sub {
      false \psk@arrowinset \psk@arrowlength \psk@arrowsize \tx@Arrow
      0 h a sub T
    } repeat
    gsave
    newpath
    false \psk@arrowinset \psk@arrowlength \psk@arrowsize \tx@Arrow
    CP
    grestore
    moveto
}
%
\@namedef{psas@<<}{%
    true \psk@arrowinset \psk@arrowlength \psk@arrowsize \tx@Arrow
    0 h neg a add T
  \psk@nArrowsB\space 2 sub {
    false \psk@arrowinset \psk@arrowlength \psk@arrowsize \tx@Arrow
    0 h neg a add T
  } repeat
  false \psk@arrowinset \psk@arrowlength \psk@arrowsize \tx@Arrow
  0 h a 5 mul 2 div sub moveto
}
%
% DG addition begin - Dec. 18/19, 1997 and Oct. 11, 2002
% Adapted from \psset@arrows
\define@key[psset]{pstricks-add}{ArrowInside}{%
  \def\pst@tempa{#1}
  \ifx\pst@tempa\@empty\def\psk@ArrowInside{}
  \else
    \begingroup
      \pst@activearrows
      \xdef\pst@tempg{<#1}%
    \endgroup
    \expandafter\psset@@ArrowInside\pst@tempg\@empty-\@empty\@nil
    \if@pst\else\@pstrickserr{Bad intermediate arrow specification: #1}\@ehpa\fi%
  \fi
}
% Adapted from \psset@@arrows
\def\psset@@ArrowInside#1-#2\@empty#3\@nil{%
  \@psttrue
  \def\next##1,#1-##2,##3\@nil{\def\pst@tempg{##2}}%
  \expandafter\next\pst@arrowtable,#1-#1,\@nil
  \@ifundefined{psas@#2}%
    {\@pstfalse\def\psk@ArrowInside{}}%
    {\def\psk@ArrowInside{#2}}%
}
% Default value empty
\psset{ArrowInside={}}
% Modified version of \pst@addarrowdef
\def\pst@addarrowdef{%
    \addto@pscode{%
	/ArrowA {
    	    \ifx\psk@arrowA\@empty
		\pst@oplineto
	    \else
		\pst@arrowdef{A}
		moveto
	    \fi
	} def
	/ArrowB {
	    \ifx\psk@arrowB\@empty \else \pst@arrowdef{B} \fi
	} def
% DG addition
	/ArrowInside {
	    \ifx\psk@ArrowInside\@empty \else \pst@arrowdefA{Inside} \fi
	} def
    }%
}
% Adapted from \pst@arrowdef
\def\pst@arrowdefA#1{%
  \ifnum\pst@repeatarrowsflag>\z@
    /Arrow#1c [ 6 2 roll ] cvx def Arrow#1c
  \fi
  \tx@BeginArrow
  \psk@arrowscale
  \@nameuse{psas@\@nameuse{psk@Arrow#1}}
  \tx@EndArrow%
}
% ArrowInsidePos parameter (default value 0.5)
\define@key[psset]{pstricks-add}{ArrowInsidePos}{\pst@checknum{#1}\psk@ArrowInsidePos}%
\psset{ArrowInsidePos=0.5}
%
% Modified version of \begin@ClosedObj
\def\begin@ClosedObj{%
    \leavevmode
    \pst@killglue
    \begingroup
	\use@par
	\solid@star
	\ifpsdoubleline \pst@setdoublesep \fi
	\pst@addarrowdef% DG addition
	\init@pscode
}
%
% Redefinition of the PostScript /Line macro to print the intermediate
% arrow on each segment of the line
%
\define@key[psset]{pstricks-add}{ArrowInsideNo}{\pst@checknum{#1}\psk@ArrowInsideNo}% hv 20031001
\define@key[psset]{pstricks-add}{ArrowInsideOffset}{\pst@checknum{#1}\psk@ArrowInsideOffset}% hv 20031001
\psset{ArrowInsideNo=1,ArrowInsideOffset=0}
%
\pst@def{Line}<{%
  NArray n 0 eq not { n 1 eq { 0 0 /n 2 def } if
  (\psk@ArrowInside) length 0 gt {
    2 copy /y1 ED /x1 ED ArrowA x1 y1
    /n n 1 sub def
    n {
      4 copy
      /y1 ED /x1 ED /y2 ED /x2 ED
      x1 y1
      \psk@ArrowInsidePos\space 1 gt {
        /Alpha y2 y1 sub x2 x1 sub atan def
        /ArrowPos \psk@ArrowInsideOffset\space def
        /Length x2 x1 sub y2 y1 sub Pyth def
        /dArrowPos \psk@ArrowInsidePos\space abs def
        {
          /ArrowPos ArrowPos dArrowPos add def
          ArrowPos Length gt { exit } if
          x1 Alpha cos ArrowPos mul add
          y1 Alpha sin ArrowPos mul add
          ArrowInside
          pop pop
        } loop
      }{
        /ArrowPos \psk@ArrowInsideOffset\space def
        /dArrowPos \psk@ArrowInsideNo 1 gt {%
          1.0 \psk@ArrowInsideNo 1.0 add div
        }{\psk@ArrowInsidePos } ifelse def
          \psk@ArrowInsideNo\space cvi {
            /ArrowPos ArrowPos dArrowPos add def
            x2 x1 sub ArrowPos mul x1 add
            y2 y1 sub ArrowPos mul y1 add
            ArrowInside
            pop pop
          } repeat
      } ifelse
      pop pop Lineto
    } repeat
  }{ ArrowA /n n 2 sub def n { Lineto } repeat } ifelse
  CP 4 2 roll ArrowB L pop pop } if%
}>
%
% Redefinition of the PostScript /Polygon macro to print the intermediate
% arrow on each segment of the line
\pst@def{Polygon}<{%
    NArray n 2 eq { 0 0 /n 3 def } if
    n 3 lt {
	n { pop pop } repeat
    }{
	n 3 gt { CheckClosed } if
	n 2 mul
	-2 roll
	/y0 ED
	/x0 ED
    	/y1 ED
    	/x1 ED
    	/xx1 x1 def
    	/yy1 y1 def
    	x1 y1
    	/x1 x0 x1 add 2 div def
    	/y1 y0 y1 add 2 div def
    	x1 y1 moveto
    	/n n 2 sub def
	/drawArrows {
	    x11 y11
	    \psk@ArrowInsidePos\space 1 gt {
		/Alpha y12 y11 sub x12 x11 sub atan def
		/ArrowPos \psk@ArrowInsideOffset\space def
		/Length x12 x11 sub y12 y11 sub Pyth def
		/dArrowPos \psk@ArrowInsidePos\space abs def
		{
		    /ArrowPos ArrowPos dArrowPos add def
		    ArrowPos Length gt { exit } if
		    x11 Alpha cos ArrowPos mul add
		    y11 Alpha sin ArrowPos mul add
		    ArrowInside
		    pop pop
		} loop
	    }{
		/ArrowPos \psk@ArrowInsideOffset\space def
		/dArrowPos \psk@ArrowInsideNo 1 gt {%
	    	    1.0 \psk@ArrowInsideNo 1.0 add div
		}{ \psk@ArrowInsidePos } ifelse def
		\psk@ArrowInsideNo cvi {
		    /ArrowPos ArrowPos dArrowPos add def
		    x12 x11 sub ArrowPos mul x11 add
		    y12 y11 sub ArrowPos mul y11 add
        	    ArrowInside
		    pop pop
		} repeat
	    } ifelse
	    pop pop Lineto
	} def
	n {
	    4 copy
	    /y11 ED /x11 ED /y12 ED /x12 ED
	    drawArrows
	} repeat
	x1 y1 x0 y0
	6 4 roll
	2 copy
	/y11 ED /x11 ED /y12 y0 def /x12 x0 def
	drawArrows
	/y11 y0 def /x11 x0 def /y12 yy1 def /x12 xx1 def
	drawArrows
	pop pop
    	closepath
    } ifelse %
}>
%
%
% Redefinition of the PostScript /OpenBezier macro to print the intermediate
% arrow
\pst@def{OpenBezier}<{%
  /dArrowPos \psk@ArrowInsideNo 1 gt {%
    1.0 \psk@ArrowInsideNo 1.0 add div
    }{ \psk@ArrowInsidePos } ifelse def
      BezierNArray
      n 1 eq { pop pop
      }{ 2 copy
        /y0 ED /x0 ED
        ArrowA
        n 4 sub 3 idiv { 6 2 roll 4 2 roll curveto } repeat
        6 2 roll
        4 2 roll
        ArrowB
        /y3 ED /x3 ED /y2 ED /x2 ED /y1 ED /x1 ED
        /cx x1 x0 sub 3 mul def
        /cy y1 y0 sub 3 mul def
        /bx x2 x1 sub 3 mul cx sub def
        /by y2 y1 sub 3 mul cy sub def
        /ax x3 x0 sub cx sub bx sub def
        /ay y3 y0 sub cy sub by sub def
        /getValues {
          ax t0 3 exp mul bx t0 t0 mul mul add cx t0 mul add x0 add
          ay t0 3 exp mul by t0 t0 mul mul add cy t0 mul add y0 add
          ax t 3 exp mul bx t t mul mul add cx t mul add x0 add
          ay t 3 exp mul by t t mul mul add cy t mul add y0 add
        } def
        /getdL {
          getValues
          3 -1 roll sub 3 1 roll sub Pyth
        } def
        /CurveLength {
          /u 0 def
          /du 0.01 def
          0 100 {
            /t0 u def
            /u u du add def
            /t u def
            getdL add
          } repeat } def
          /GetArrowPos {
            /ende \psk@ArrowInsidePos\space 1 gt
              {ArrowPos}
              {ArrowPos CurveLength mul} ifelse def
            /u 0 def
            /du 0.01 def
            /sum 0 def
            { /t0 u def
              /u u du add def
              /t u def
              /sum getdL sum add def
              sum ende gt {exit} if
            } loop u
          } def
          /ArrowPos \psk@ArrowInsideOffset\space def
          /loopNo \psk@ArrowInsidePos\space 1 gt {%
            CurveLength \psk@ArrowInsidePos\space div cvi
          }{ \psk@ArrowInsideNo } ifelse def
            loopNo cvi {
              /ArrowPos ArrowPos dArrowPos add def
              /t GetArrowPos def
              /t0 t 0.95 mul def
              getValues
              ArrowInside pop pop pop pop
            } repeat
            x1 y1 x2 y2 x3 y3 curveto
  } ifelse
}>
%
% Redefinition of the PostScript /NCLine macro to print the intermediate
% arrow of the line
\pst@def{NCLine}<{%
	NCCoor
	tx@Dict begin
	ArrowA CP 4 2 roll ArrowB
	4 copy
	/y2 ED /x2 ED /y1 ED /x1 ED
	x1 y1
	\psk@ArrowInsidePos\space 1 gt {
		/Alpha y2 y1 sub x2 x1 sub atan def
		/ArrowPos \psk@ArrowInsideOffset\space def
		/Length x2 x1 sub y2 y1 sub Pyth def
		/dArrowPos \psk@ArrowInsidePos\space abs def
		{%
			/ArrowPos ArrowPos dArrowPos add def
			ArrowPos Length gt { exit } if
			x1 Alpha cos ArrowPos mul add
			y1 Alpha sin ArrowPos mul add
			ArrowInside
			pop pop
		} loop
	}{%
		/ArrowPos \psk@ArrowInsideOffset\space def
		/dArrowPos \psk@ArrowInsideNo 1 gt {%
			1.0 \psk@ArrowInsideNo 1.0 add div
		}{ \psk@ArrowInsidePos } ifelse def
		\psk@ArrowInsideNo\space cvi {
			/ArrowPos ArrowPos dArrowPos add def
			x2 x1 sub ArrowPos mul x1 add
			y2 y1 sub ArrowPos mul y1 add
			ArrowInside
			pop pop
		} repeat
	} ifelse
	pop pop lineto pop pop
	end%
}>
%
\pst@def{NCCurve}<{%
	GetEdgeA GetEdgeB
	xA1 xB1 sub yA1 yB1 sub
	Pyth 2 div dup 3 -1 roll mul
	/ArmA ED
	mul
	/ArmB ED
	/ArmTypeA 0 def
	/ArmTypeB 0 def
	GetArmA GetArmB
	xA2 yA2 xA1 yA1
	2 copy
	/y0 ED /x0 ED
	tx@Dict begin
		ArrowA
	end
	xB2 yB2 xB1 yB1
	tx@Dict begin
		ArrowB
	end
	/y3 ED /x3 ED /y2 ED /x2 ED /y1 ED /x1 ED
	/cx x1 x0 sub 3 mul def
	/cy y1 y0 sub 3 mul def
	/bx x2 x1 sub 3 mul cx sub def
	/by y2 y1 sub 3 mul cy sub def
	/ax x3 x0 sub cx sub bx sub def
	/ay y3 y0 sub cy sub by sub def
	/getValues {
		ax t0 3 exp mul bx t0 t0 mul mul add cx t0 mul add x0 add
		ay t0 3 exp mul by t0 t0 mul mul add cy t0 mul add y0 add
		ax t 3 exp mul bx t t mul mul add cx t mul add x0 add
	ay t 3 exp mul by t t mul mul add cy t mul add y0 add
	} def
	/getdL {
		getValues
		3 -1 roll sub 3 1 roll sub Pyth
	} def
	/CurveLength {
		/u 0 def
		/du 0.01 def
		0 100 {
			/t0 u def
			/u u du add def
			/t u def
			getdL add
		} repeat } def
	/GetArrowPos {
		/ende \psk@ArrowInsidePos\space 1 gt {ArrowPos}{ArrowPos CurveLength mul} ifelse def
		/u 0 def
		/du 0.01 def
		/sum 0 def
		{
			/t0 u def
			/u u du add def
			/t u def
			/sum getdL sum add def
			sum ende gt {exit} if
		} loop u
	} def
	/dArrowPos \psk@ArrowInsideNo 1 gt {%
		1.0 \psk@ArrowInsideNo 1.0 add div
	}{ \psk@ArrowInsidePos } ifelse def
	/ArrowPos \psk@ArrowInsideOffset\space def
	/loopNo \psk@ArrowInsidePos\space 1 gt {%
		CurveLength \psk@ArrowInsidePos\space div cvi
		}{ \psk@ArrowInsideNo } ifelse def
	loopNo cvi {
		/ArrowPos ArrowPos dArrowPos add def
		/t GetArrowPos def
		/t0 t 0.95 mul def
		getValues
		ArrowInside pop pop pop pop
	} repeat
	x1 y1 x2 y2 x3 y3 curveto
	/LPutVar [ xA1 yA1 xA2 yA2 xB2 yB2 xB1 yB1 ] cvx def
	/LPutPos { t LPutVar BezierMidpoint } def
	/HPutPos { { HPutLines } HPutCurve } def
	/VPutPos { { VPutLines } HPutCurve } def
}>
%
%\def\psDashColorI{black}
%\def\psDashColorII{red}
\define@key[psset]{pstricks-add}{dashColorI}{\pst@getcolor{#1}\psDashColorI}
\define@key[psset]{pstricks-add}{dashColorII}{\pst@getcolor{#1}\psDashColorII}
\define@key[psset]{pstricks-add}{dashNo}{\edef\psk@dashNo{#1}}
%
\define@key[psset]{pstricks-add}{linecap}{\edef\psk@linecap{#1}}
%\psset{dashColorI=black,dashColorII=red,dashNo=0.2,linecap=0}
\psset{dashColorI=black,dashColorII=red,dashNo=0.2,linecap=0}
%
\pst@def{LineII}<{%
	NArray
	/n n 1 sub def
	/y1 ED /x1 ED x1 y1 ArrowA x1 y1 moveto
	\psk@linecap\space 3 gt
		{\psk@linecap\space 0 lt {0 setlinecap} if }
		{\psk@linecap\space setlinecap} ifelse
	n {
		/y2 ED /x2 ED
		/y0 y1 def /x0 x1 def
		/length x2 x1 sub y2 y1 sub Pyth def
		\psk@dashNo\space 1.0 lt
			{/cntMax 1.0 \psk@dashNo\space div .49 add cvi def}
			{/cntMax length \psk@dashNo\space div .49 add cvi def} ifelse
		x2 x1 sub cntMax div /dx ED
		y2 y1 sub cntMax div /dy ED
		/cnt 0 def
		cntMax {
			gsave
			/x1 x1 dx add def
			/y1 y1 dy add def
			x1 y1
			cnt 2 mod 0 eq
				{ \pst@usecolor\psDashColorI }
				{ \pst@usecolor\psDashColorII } ifelse
			lineto stroke
			/cnt cnt 1 add def
			grestore
			x1 y1 moveto
		} repeat
		/y1 y2 def /x1 x2 def
	} repeat
	x0 y0 x2 y2 ArrowB L pop pop
}>
%
\def\pslineII{\pst@object{pslineII}}
\def\pslineII@i{%
    \pst@getarrows{%
        \begin@OpenObj
        \pst@getcoors[\pslineII@ii%
    }%
}
\def\pslineII@ii{%
	\addto@pscode{%
		\pst@cp
		\ifshowpoints true \else false \fi
		\tx@LineII
	}%
	\end@OpenObj%
}
%
\newdimen\pswBegin\newdimen\pswEnd
\define@key[psset]{pstricks-add}{wBegin}{\pssetlength\pswBegin{#1}}
\define@key[psset]{pstricks-add}{wEnd}{\pssetlength\pswEnd{#1}}
\psset{wBegin=\pslinewidth, wEnd=\pslinewidth}
%
\pst@def{LineIII}<{%
	NArray
	/n n 1 sub def
	/YA ED /XA ED
	/bA2 \pst@number\pswBegin\space 2.0 div def
	/bB2 \pst@number\pswEnd\space 2.0 div def
	/xStep bB2 bA2 sub n div def
	0 setlinecap
	n {
		XA YA moveto
		/YB ED /XB ED
		/Alpha YB YA sub XB XA sub atan def
		/Alphasin Alpha sin def
		/Alphacos Alpha cos def
        \pst@usecolor\pslinecolor
		0.01 SLW
		/bB2 bA2 xStep add def
		XA Alphasin bA2 mul sub
		YA Alphacos bA2 mul add lineto
		XB Alphasin bB2 mul sub
		YB Alphacos bB2 mul add lineto
		XB Alphasin bB2 mul add
		YB Alphacos bB2 mul sub lineto
		XA Alphasin bA2 mul add
		YA Alphacos bA2 mul sub lineto
		XA YA lineto fill
		/XA XB def /YA YB def
		/bA2 bB2 def
	} repeat
}>
%
\def\pslineIII{\pst@object{pslineIII}}
\def\pslineIII@i{%
  \pst@getarrows{%
     \begin@OpenObj
       \pst@getcoors[\pslineIII@ii%
  }%
}
\def\pslineIII@ii{%
  \addto@pscode{%
    \pst@cp
    \ifshowpoints true \else false \fi
    \tx@LineIII
  }%
  \end@OpenObj%
}
%
\define@key[psset]{pstricks-add}{randomPoints}{\def\psk@randomPoints{#1}}
\newif\ifPst@randomColor
\define@key[psset]{pstricks-add}{color}[true]{\@nameuse{Pst@randomColor#1}}
\psset{randomPoints=1000,color=false}
%
\def\psRandom{\pst@object{psRandom}}%  hv  2004-11-12
\def\psRandom@i{\@ifnextchar({\psRandom@ii}{\psRandom@iii(0,0)(1,1)}}
\def\psRandom@ii(#1){\@ifnextchar({\psRandom@iii(#1)}{\psRandom@iii(0,0)(#1)}}
\def\psRandom@iii(#1)(#2)#3{%
  \def\pst@tempa{#3}%
  \ifx\pst@tempa\pst@empty\psclip{\psframe(#2)}\else\psclip{#3}\fi
  \pst@getcoor{#1}\pst@tempa 
  \pst@getcoor{#2}\pst@tempb 
  \begin@SpecialObj
  \addto@pscode{
    \pst@tempa\space /yMin exch def 
    /xMin exch def
    \pst@tempb\space /yMax exch def 
    /xMax exch def 
    /dy yMax yMin sub def
    /dx xMax xMin sub def
    rrand srand                 % initializes the random generator
    /getRandReal { rand 2147483647 div } def
    \psk@dotsize % defines /DS ... def
    \@nameuse{psds@\psk@dotstyle}
    \psk@randomPoints {
     \ifPst@randomColor getRandReal getRandReal getRandReal setrgbcolor \fi
     getRandReal dx mul xMin add
     getRandReal dy mul yMin add
     Dot
     \ifx\psk@fillstyle\psfs@solid fill \fi stroke
    } repeat
  }%
  \end@SpecialObj
  \endpsclip
  \ignorespaces
}
%
%%%%%%%%%%%%%%%%%%%%%%%%%%%%%%%%%%%%%%%%%%%%%%%%%%%%%%%%%%%%%%%%%%%%%%%
%%%%%%%%%%%                                                  %%%%%%%%%%
%%%%%%%%%%%                pst-node                          %%%%%%%%%%
%%%%%%%%%%%                                                  %%%%%%%%%%
%%%%%%%%%%%%%%%%%%%%%%%%%%%%%%%%%%%%%%%%%%%%%%%%%%%%%%%%%%%%%%%%%%%%%%%
%
% This fixes a bug in pst-node, where the XY-direction is wrong
% the types are changed 1<->2 between X<->Y
%
\define@key[psset]{pstricks-add}{XnodesepA}{%
    \pst@getlength{#1}\psk@nodesepA
    \def\psk@nodeseptypeA{2 }%
}
\define@key[psset]{pstricks-add}{XnodesepB}{%
    \pst@getlength{#1}\psk@nodesepB
    \def\psk@nodeseptypeB{2 }%
}
\define@key[psset]{pstricks-add}{Xnodesep}{%
    \pst@getlength{#1}\psk@nodesepA
    \let\psk@nodesepB\psk@nodesepA
    \def\psk@nodeseptypeA{2 }%
    \def\psk@nodeseptypeB{2 }%
}
\define@key[psset]{pstricks-add}{YnodesepA}{%
    \pst@getlength{#1}\psk@nodesepA
    \def\psk@nodeseptypeA{1 }%
}
\define@key[psset]{pstricks-add}{YnodesepB}{%
    \pst@getlength{#1}\psk@nodesepB
    \def\psk@nodeseptypeB{1 }%
}
\define@key[psset]{pstricks-add}{Ynodesep}{%
    \pst@getlength{#1}\psk@nodesepA
    \let\psk@nodesepB\psk@nodesepA
    \def\psk@nodeseptypeA{1 }%
    \def\psk@nodeseptypeB{1 }%
}
%
\pst@def{NCLineII}<{
  NCCoor
  /y1 ED /x1 ED x1 y1 ArrowA x1 y1 moveto
  /y2 ED /x2 ED
  /y0 y1 def /x0 x1 def
  /length x2 x1 sub y2 y1 sub Pyth def
  \psk@dashNo\space 1.0 lt
     {/cntMax 1.0 \psk@dashNo\space div .49 add cvi def}
     {/cntMax length \psk@dashNo\space div .49 add cvi def} ifelse
  x2 x1 sub cntMax div /dx ED
  y2 y1 sub cntMax div /dy ED
  /cnt 0 def
  cntMax {
    gsave
    /x1 x1 dx add def
    /y1 y1 dy add def
    x1 y1
    cnt 2 mod 0 eq
      { \pst@usecolor\psDashColorI }
      { \pst@usecolor\psDashColorII } ifelse
    lineto stroke
    /cnt cnt 1 add def
    grestore
    x1 y1 moveto
  } repeat
  x0 y0 x2 y2 ArrowB L pop pop%
}>
%
\def\nclineII{\pst@object{nclineII}}%
\def\nclineII@i{\check@arrow{\nclineII@ii}}%
\def\nclineII@ii#1#2{\nc@object{Open}{#1}{#2}{.5}%
    {\tx@NCLineII /LPutPos { xB yB xA yA \tx@LPutLine } def}%
}%
\def\pclineII{\pst@object{pclineII}}%
\def\pclineII@i{\pc@object\nclineII@ii}%
%
\define@key[psset]{pstricks-add}{lineAngle}{\psset{armB=0.5}\edef\psk@lineAngle{#1}}%
\psset{lineAngle=0}%
%
\pst@def{NCDiag}<{
  GetEdgeA GetEdgeB GetArmA GetArmB mark
  \psk@lineAngle\space abs 0 gt {
    /xTemp xA2 10 add def
    /yTemp yA2 \psk@lineAngle\space dup sin exch cos div 10 mul add def
    /dY1 yTemp yA2 sub def
    /dX1 xTemp xA2 sub def
    /dY2 yB2 yB1 sub def
    /dX2 xB2 xB1 sub def
    dX1 abs 0.01 lt {
      /m2 dY2 dX2 div def
      /xB2 xA2 def
      /yB2 xA2 xB1 sub m2 mul yB1 add def
    }{
      dX2 abs 0.01 lt {
        /m1 dY1 dX1 div def
        /xB2 xB1 def
        /yB2 xB1 xA2 sub m1 mul yA2 add def
      }{%
        /m1 dY1 dX1 div def
        /m2 dY2 dX2 div def
        /xB2 m1 xA2 mul m2 xB1 mul sub yA2 sub yB1 add m1 m2 sub div def
        /yB2 xB2 xA2 sub m1 mul yA2 add def
      } ifelse
    } ifelse
  } if
  ArmB 0 ne { xB1 yB1 } if
  xB2 yB2
  xA2 yA2
  ArmA 0 ne { xA1 yA1 } if
  tx@Dict begin false Line end
  /LPutVar [ xB1 yB1 xB2 yB2 xA2 yA2 xA1 yA1 ] cvx def
  /LPutPos { LPutLines } def
  /HPutPos { HPutLines } def
  /VPutPos { VPutLines } def
}>
% hv 2003-12-22
\pst@def{NCDiagg}<{
  GetEdgeA GetArmA \psk@lineAngle\space abs 0 gt { \psk@lineAngle\space }
    { yB yA2 sub xB xA2 sub Atan 180 add } ifelse /AngleB ED
  GetEdgeB mark
  \psk@lineAngle\space abs 0 gt {
    /dY2 yA2 yA1 sub def
    /dX2 xA2 xA1 sub def
    \psk@lineAngle\space abs 90 eq {
      /m2 dY2 dX2 div def
      /yA2 xB xA2 sub m2 mul yA2 add def
      /xA2 xB def
    }{
      /m1 \psk@lineAngle\space dup sin exch cos div def % tan alpha
      dX2 abs 0.01 lt {
        /yA2 xA1 xB sub m1 mul yB add def
        /xA2 xA1 def
      }{%
        /m2 dY2 dX2 div def
        /xA2 m1 xB mul m2 xA2 mul sub yA2 add yB sub m1 m2 sub div def
        /yA2 xA2 xB sub m1 mul yB add def
      } ifelse
    } ifelse
  } if
  xB1 yB1 xA2 yA2
  ArmA 0 ne { xA1 yA1 } if
  tx@Dict begin false Line end
  /LPutVar [ xB1 yB1 xA2 yA2 xA1 yA1 ] cvx def
  /LPutPos { LPutLines } def
  /HPutPos { HPutLines } def
  /VPutPos { VPutLines } def
}>
%
\def\ncbarr{\pst@object{ncbarr}}
\def\ncbarr@i#1#2{
  \begingroup
  \use@par%
  \setLNode(#1)(#2){0.5}{barr@tempNode}%
  \pst@dimm=\psk@angleA pt
  \pst@dimn=180pt
  % be sure, that angleA is 0 or 180. if not, we set it to 0
  \ifdim\pst@dimm=\z@\else\ifdim\pst@dimm=\pst@dimn\else\psset{angleA=0}\fi\fi
  \ncbar[arrows=-]{#1}{barr@tempNode}
  \ifdim\psk@angleA pt=\z@\relax
    \ncbar[angleA=180,angleB=180]{barr@tempNode}{#2}
  \else\ncbar[angleA=0,angleB=0]{barr@tempNode}{#2}\fi%
  \endgroup
}
%   #1-------#4----------------#2
% where #1#4= #3 * #1#2
%
\def\setLNode(#1)(#2)#3#4{%
  \pst@getcoor{#1}\pst@tempa%
  \pst@getcoor{#2}\pst@tempb%
  \pnode(!%
    \pst@tempa /YA exch \pst@number\psyunit div def
    /XA exch \pst@number\psxunit div def
    \pst@tempb /YB exch \pst@number\psyunit div def
    /XB exch \pst@number\psxunit div def
    /dx XB XA sub def
    /dy YB YA sub def
    XA dx #3\space mul add YA dy #3\space mul add){#4}
}
%
% build the linear combination #2*#1+#4*#3=#5
\def\setLCNode(#1)#2(#3)#4#5{%
  \pst@getcoor{#1}\pst@tempa%
  \pst@getcoor{#3}\pst@tempb%
  \pnode(!%
    \pst@tempa /YA exch \pst@number\psyunit div def
    /XA exch \pst@number\psxunit div def
    \pst@tempb /YB exch \pst@number\psyunit div def
    /XB exch \pst@number\psxunit div def
    XA #2\space mul XB #4\space mul add
    YA #2\space mul YB #4\space mul add){#5}
}
%
\newif\ifPst@trueAngle
\define@key[psset]{pstricks-add}{trueAngle}[true]{\@nameuse{Pst@trueAngle#1}}
\psset{trueAngle=false}
%
\def\psRelLine{\pst@object{psRelLine}}
\def\psRelLine@i{\@ifnextchar({\psRelLine@iii}{\psRelLine@ii}}
\def\psRelLine@ii#1{%
  \addto@par{arrows=#1}%
  \psRelLine@iii%
}
\def\psRelLine@iii(#1)(#2)#3#4{{
  \pst@killglue
  \use@par
  \pst@getcoor{#1}\pst@tempa
  \pst@getcoor{#2}\pst@tempb
  \pnode(!
    \pst@tempa /YA exch \pst@number\psyunit div def
    /XA exch \pst@number\psxunit div def
    \pst@tempb /YB exch \pst@number\psyunit div def
    /XB exch \pst@number\psxunit div def
    /AlphaStrich \psk@angleA\space def
    /unit \pst@number\psyunit \pst@number\psxunit div def % yunit/xunit
%            
    /dx XB XA sub  def
    /dy YB YA sub \ifPst@trueAngle\space unit mul \fi\space def
    /laenge dy dup mul dx dup mul add sqrt #3 mul def
    /Alpha dy dx atan def 
    /beta Alpha AlphaStrich add def
    laenge beta cos mul XA add
    laenge beta sin mul \ifPst@trueAngle\space unit div \fi\space YA add ){#4}%
  \psline(#1)(#4)%
}\ignorespaces}
%
% #1 options
% draw a parallel line to #2 #3
%     #2---------#3
%         #4----------#5(new node)
% #5 length of the line
% #6 node name
\def\psParallelLine{\pst@object{psParallelLine}}
\def\psParallelLine@i{\@ifnextchar({\psParallelLine@iii}{\psParallelLine@ii}}
\def\psParallelLine@ii#1{%
  \addto@par{arrows=#1}%
  \psParallelLine@iii%
}
\def\psParallelLine@iii(#1)(#2)(#3)#4#5{{
  \pst@killglue
  \use@par
  \pst@getcoor{#1}\pst@tempa
  \pst@getcoor{#2}\pst@tempb
  \pst@getcoor{#3}\pst@tempc
%  \pst@getlength{#4}\pst@dima
  \pnode(!%
     \pst@tempa /YA exch \pst@number\psyunit div def
     /XA exch \pst@number\psxunit div def
     \pst@tempb /YB exch \pst@number\psyunit div def
     /XB exch \pst@number\psxunit div def
     \pst@tempc /YC exch \pst@number\psyunit div def
     /XC exch \pst@number\psxunit div def
%            
    /dx XB XA sub  def
    /dy YB YA sub  def
    /laenge dy dup mul dx dup mul add sqrt #4 mul def
    /Alpha dy dx atan def 
    laenge Alpha cos mul XC add
    laenge Alpha sin mul YC add ){#5}%
  \psline(#3)(#5)
}\ignorespaces}
%
\def\psIntersectionPoint(#1)(#2)(#3)(#4)#5{%
    \pst@getcoor{#1}\pst@tempa
    \pst@getcoor{#2}\pst@tempb
    \pst@getcoor{#3}\pst@tempc
    \pst@getcoor{#4}\pst@tempd
\pnode(!%
     \pst@tempa /YA exch \pst@number\psyunit div def
     /XA exch \pst@number\psxunit div def
     \pst@tempb /YB exch \pst@number\psyunit div def
     /XB exch \pst@number\psxunit div def
     \pst@tempc /YC exch \pst@number\psyunit div def
     /XC exch \pst@number\psxunit div def
     \pst@tempd /YD exch \pst@number\psyunit div def
     /XD exch \pst@number\psxunit div def
    /dY1 YB YA sub def
    /dX1 XB XA sub def
    /dY2 YD YC sub def
    /dX2 XD XC sub def
    dX1 abs 0.01 lt {
        /m2 dY2 dX2 div def
        XA dup XC sub m2 mul YC add
    }{
        dX2 abs 0.01 lt {
            /m1 dY1 dX1 div def
            XC dup XA sub m1 mul YA add
        }{%
            /m1 dY1 dX1 div def
            /m2 dY2 dX2 div def
            m1 XA mul m2 XC mul sub YA sub YC add m1 m2 sub div dup
            XA sub m1 mul YA add
        } ifelse
    } ifelse ){#5}%
}
%
%
%%%%%%%%%%%%%%%%%%%%%%%%%%%%%%%%%%%%%%%%%%%%%%%%%%%%%%%%%%%%%%%%%%%%%%%
%%%%%%%%%%%                                                  %%%%%%%%%%
%%%%%%%%%%%                pst-plot                          %%%%%%%%%%
%%%%%%%%%%%                                                  %%%%%%%%%%
%%%%%%%%%%%%%%%%%%%%%%%%%%%%%%%%%%%%%%%%%%%%%%%%%%%%%%%%%%%%%%%%%%%%%%%
%
\newif\ifPst@plot@comma
\define@key[psset]{pstricks-add}{comma}[true]{\@nameuse{Pst@plot@comma#1}}
%
\newif\ifPst@plot@xAxis
\newif\ifPst@plot@yAxis
\newif\ifPst@plot@xyAxes
\define@key[psset]{pstricks-add}{xAxis}[true]{\@nameuse{Pst@plot@xAxis#1}}
\define@key[psset]{pstricks-add}{yAxis}[true]{\@nameuse{Pst@plot@yAxis#1}}
\define@key[psset]{pstricks-add}{xyAxes}[true]{\psset{xAxis=true,yAxis=true}}%
\psset{xyAxes=true}
%
\define@key[psset]{pstricks-add}{xyDecimals}{\def\psk@xDecimals{#1}\def\psk@yDecimals{#1}}
\define@key[psset]{pstricks-add}{xDecimals}{\def\psk@xDecimals{#1}}
\define@key[psset]{pstricks-add}{yDecimals}{\def\psk@yDecimals{#1}}
\psset{xyDecimals={}}%
%
\define@key[psset]{pstricks-add}{xyLabel}{\def\psk@xLabel{#1}\def\psk@yLabel{#1}}
\define@key[psset]{pstricks-add}{xLabel}{\def\psk@xLabel{#1}}
\define@key[psset]{pstricks-add}{yLabel}{\def\psk@yLabel{#1}}
\psset{xyLabel={}}
%
\define@key[psset]{pstricks-add}{xlogBase}{\edef\psk@xlogBase{#1}}
\define@key[psset]{pstricks-add}{ylogBase}{\edef\psk@ylogBase{#1}}
\define@key[psset]{pstricks-add}{xylogBase}{\edef\psk@xlogBase{#1}\edef\psk@ylogBase{#1}}%
\psset{xylogBase={}}
%
%logLines=all|x|y|none  (0,1,2,3)
\define@key[psset]{pstricks-add}{logLines}{\pst@expandafter\psset@@logLines{#1}\@nil\psk@logLines}
\def\psset@@logLines#1#2\@nil#3{%
  \ifx#1a\let#3\z@\else
    \ifx#1x\let#3\@ne\else
      \ifx#1y\let#3\tw@\else
        \ifx#1n\let#3\thr@@\else
          \@pstrickserr{Bad argument: `#1#2'}\@ehpa
  \fi\fi\fi\fi}
\psset{logLines=none}
%
% Define "ticklines" parameter (ticklines=all|x|y|none with default=none)
%\def\psset@ticklines#1{\pst@expandafter\psset@@ticklines{#1}\@nil\psk@ticklines}
%\psset@ticklines{none}
\define@key[psset]{pstricks-add}{ylabelFactor}{\edef\psk@ylabelFactor{#1}}
\define@key[psset]{pstricks-add}{xlabelFactor}{\edef\psk@xlabelFactor{#1}}
\psset{xlabelFactor=\relax,ylabelFactor=\relax}
% The parameter styles for the tick lines
% (default=arrows=-,linestyle=dotted,dotsep=5pt)
%\newpsstyle{ticklinesXStyle}{arrows=-,linestyle=dotted,dotsep=5pt}
%\newpsstyle{ticklinesYStyle}{arrows=-,linestyle=dotted,dotsep=5pt}
%
%
%% #1 integer
%% #2 decimals
%% #3 dot
\def\@stripDecimals#1.#2.#3\@nil{%
  \def\dummy{#1}%
  \ifx\dummy\@empty\the\@zero\else#1\fi% the integer part
}
%
\def\pst@@@hlabel#1{%
  \edef\@xyDecimals{\psk@xDecimals}%
  \psk@xLabel%
  \ifnum\psk@ticks<\tw@ % ticks=all|x
    \ifx\psk@xlogBase\@empty
      \ifPst@plot@comma\pshlabel{\expandafter\@LabelComma#1..\@nil\psk@xlabelFactor}%
      \else\pshlabel{#1}%
      \fi%
    \else%
      {\psk@xLabel\pshlabel{\psk@xlogBase\textsuperscript{\expandafter\@stripDecimals#1..\@nil }}}
    \fi%
  \fi%
}
\def\pst@@@vlabel#1{%
  \edef\@xyDecimals{\psk@yDecimals}%
  \psk@yLabel%
  \ifodd\psk@ticks  % ticks=all||y (0,2)
  \else
    \ifx\psk@ylogBase\@empty
      \ifPst@plot@comma\psvlabel{\expandafter\@LabelComma#1..\@nil\psk@ylabelFactor}%
      \else\psvlabel{#1}%
      \fi%
    \else%
      {\psk@yLabel\psvlabel{\psk@ylogBase\textsuperscript{\expandafter\@stripDecimals#1..\@nil }}}
    \fi%
  \fi%
}
\newcount\@digitcounter\@digitcounter=0\relax
\def\@inc@digitcounter{\global\advance\@digitcounter by 1\relax}
\def\@get@digitcounter{\the\@digitcounter\relax}
\def\@Reset@digitcounter{\global\@digitcounter=0\relax}
\def\@zeroFill{%
  \ifnum \@xyDecimals>\@get@digitcounter
    \bgroup
      0\@inc@digitcounter\@zeroFill
    \egroup%
  \fi%
}
% #1 the value, maybe empty
%
\def\@process@digits#1#2;{%
	\ifx *#1\@zeroFill\else#1\@inc@digitcounter\@process@digits#2;\fi%
}
\def\@writeDecimals#1{%
  \ifx\@xyDecimals\@empty% take value as is
    \def\@tempa{#1}% write only if not empty
    \ifx\@tempa\@empty% write nothing
    \else
      \ifPst@plot@comma,\else.\fi%
      #1%
    \fi%
  \else% write only \xy@decimals
    \ifnum\@xyDecimals>\@zero
      \ifPst@plot@comma,\else.\fi
        \@Reset@digitcounter
        \@process@digits#1*;
      \fi%
  \fi%
}
%% #1 integer
%% #2 decimals
%% #3 dot
\def\@LabelComma#1.#2.#3\@nil{%
  \def\dummy{#1}%
  \ifx\dummy\@empty\the\@zero\else#1\fi% the integer part
  \def\dummy{#2}%
  \ifx\dummy\@empty\@writeDecimals{}\else\@writeDecimals{#2}\fi%
}
%
\def\psxs@axes{{%
  \ifPst@plot@xAxis\psxs@@axes\pst@dima\pst@dimb\pst@dimc\pst@dimd{}\fi
  \ifPst@plot@yAxis\psxs@@axes\pst@dima\pst@dimb\pst@dimc\pst@dimd{exch}\fi
}}
\def\psaxes{\pst@object{psaxes}}
\def\psaxes@i{\pst@getarrows\psaxes@ii}
\def\psaxes@ii(#1){\@ifnextchar({\psaxes@iii(#1)}{\psaxes@iv(0,0)(0,0)(#1)}}
\def\psaxes@iii(#1)(#2){\@ifnextchar({\psaxes@iv(#1)(#2)}{\psaxes@iv(#1)(#1)(#2)}}
%
\def\psaxes@iv(#1,#2)(#3,#4)(#5,#6){%
  \pst@killglue
  \begingroup
  \setbox\pst@hbox=\hbox\bgroup%
  \use@par%
  \pssetxlength\pst@dimg{#1}% o-x
  \pssetylength\pst@dimh{#2}% o-y
  \pssetxlength\pst@dima{#3}% bl-x
  \pssetylength\pst@dimb{#4}% bl-y
  \pssetxlength\pst@dimc{#5}% ur-x
  \pssetylength\pst@dimd{#6}% ur-y
    % If minimum values are negative in log mode, we modify Ox
    % (respectively Oy) if this was not done by the user
    % X axis labels (\psk@log = 0 or 1)
  \ifx\psk@xlogBase\@empty\else%
    \ifx\psk@Ox\@empty\else% we do nothing ... hv
      \ifdim\psk@Ox pt=\z@
        \ifdim#3pt<\z@
          \pssetxlength\pst@dimg{#3}% o-x
          \psset@Ox{#3}%
	\fi
      \fi%
    \fi%
  \fi%
% Y axis labels (\psk@log = 0 or 2)
  \ifx\psk@ylogBase\@empty\else%
    \ifx\psk@Ox\@empty\else
      \ifdim#4pt<\z@
        \ifdim\psk@Oy pt=\z@
          \pssetylength\pst@dimh{#4}% o-y
          \psset@Oy{#4}%
	\fi
      \fi
    \fi
  \fi
% Whole thing will be translated to origin:
  \advance\pst@dima-\pst@dimg % Dist. from bl-x to o-x
  \advance\pst@dimb-\pst@dimh % Dist. from bl-y to o-y
  \advance\pst@dimc-\pst@dimg % Dist. from ur-x to o-x
  \advance\pst@dimd-\pst@dimh % Dist. from ur-y to o-y
% Make lines/arrows or frame:
  \@nameuse{psxs@\psk@axesstyle}%  \psxs@axes or \psxs@frame
  \ifPst@plot@yAxis  
  \begingroup
    \ifdim\pst@dima=\z@\else\showoriginfalse\fi
    \ifnum\psk@dy=\z@
      \pst@dimg=\psk@Dy\psyunit
      \edef\psk@dy{\number\pst@dimg}%
    \fi%
    \ifPst@plot@xAxis\else\showorigintrue\fi%
    \pst@vlabels{\pst@dimd}{\psk@arrowB}{#3}{#5}%
    \ifPst@plot@xAxis\showoriginfalse\fi%
    \pst@vlabels{\pst@dimb}{\psk@arrowA}{#3}{#5}%
  \endgroup%
  \fi
  \ifPst@plot@xAxis  
  \begingroup%
    \ifdim\pst@dimb=\z@\else\showoriginfalse\fi%
    \ifnum\psk@dx=\z@%
      \pst@dimg=\psk@Dx\psxunit%
      \edef\psk@dx{\number\pst@dimg}%
    \fi%
    \ifPst@plot@yAxis\else\showorigintrue\fi%
    \pst@hlabels{\pst@dimc}{\psk@arrowB}{#4}{#6}%
    \ifPst@plot@yAxis\showoriginfalse\fi%
    \pst@hlabels{\pst@dima}{\psk@arrowA}{#4}{#6}%
  \endgroup%
  \fi
% Now close "\pst@hbox" (which is 0-dimensional), and put it at the origin.
  \egroup%
  \pssetxlength\pst@dimg{#1}%
  \pssetylength\pst@dimh{#2}%
  \leavevmode\psput@cartesian\pst@hbox%
  \ignorespaces%
  \endgroup
}
%
\newdimen\psk@subticksize\psk@subticksize=\z@
\newdimen\pst@xticksizeA
\newdimen\pst@xticksizeB
\newdimen\pst@xticksizeC
\newdimen\pst@yticksizeA
\newdimen\pst@yticksizeB
\newdimen\pst@yticksizeC
%
\define@key[psset]{}{ticksize}{\psset{xticksize=#1,yticksize=#1}}
\define@key[psset]{pstricks-add}{xticksize}{%
  \pst@expandafter\pst@getydimdim{#1} {\pst@missing} {} {}\@nil%
  \pst@xticksizeA=\the\pst@dimm%
  \pst@xticksizeB=\the\pst@dimn%
  \edef\psk@xticksize{\pst@number\pst@xticksizeA\space \pst@number\pst@xticksizeB}%
  \ifdim\pst@xticksizeA=0pt\pst@xticksizeC=\z@\else
    \pst@absdim{\pst@xticksizeA}\pst@dimm%
    \pst@absdim{\pst@xticksizeB}\pst@dimn%
    \pst@maxdim{\pst@dimm}{\pst@dimn}\pst@dimo%
    \pst@xticksizeC=\pst@dimo%
  \fi%
}
\define@key[psset]{}{yticksize}{%  overwrites the definition in pstricks
  \pst@expandafter\pst@getxdimdim{#1} {\pst@missing} {} {}\@nil%
  \pst@yticksizeA=\the\pst@dimm%
  \pst@yticksizeB=\the\pst@dimn%
  \edef\psk@yticksize{\pst@number\pst@yticksizeA \pst@number\pst@yticksizeB}%
  \ifdim\pst@yticksizeA=0pt\pst@yticksizeC=\z@\else
    \pst@absdim{\pst@yticksizeA}\pst@dimm%
    \pst@absdim{\pst@yticksizeB}\pst@dimn%
    \pst@maxdim{\pst@dimm}{\pst@dimn}\pst@dimo%
    \pst@yticksizeC=\pst@dimo%
  \fi%
}
\psset{xticksize=-4pt 4pt,yticksize=-4pt 4pt}
%
% full= 0, top=1, bottom=-1
\define@key[psset]{pstricks-add}{tickstyle}{%
  \pst@expandafter\psset@@tickstyle{#1}\@nil%
  \ifnum\psk@tickstyle>\z@%
    \psset{xticksize=0 \pst@xticksizeB}%
    \psset{yticksize=0 \pst@yticksizeB}%
  \else\ifnum\psk@tickstyle<\z@%
%    \expandafter\psset@xticksize\expandafter{\pst@xticksizeA\space 0}%
%    \expandafter\psset@yticksize\expandafter{\pst@yticksizeA\space 0}%
    \psset{xticksize=\pst@xticksizeA\space 0}%
    \psset{yticksize=\pst@yticksizeA\space 0}%
  \fi\fi%
}
\psset[pstricks-add]{tickstyle=full}
%
%
\define@key[psset]{pstricks-add}{subticks}{\edef\psk@xsubticks{#1}\edef\psk@ysubticks{#1}}
\define@key[psset]{pstricks-add}{xsubticks}{\edef\psk@xsubticks{#1}}
\define@key[psset]{pstricks-add}{ysubticks}{\edef\psk@ysubticks{#1}}
%
\define@key[psset]{pstricks-add}{subticksize}{\edef\psk@xsubticksize{#1}\edef\psk@ysubticksize{#1}}
\define@key[psset]{pstricks-add}{xsubticksize}{\edef\psk@xsubticksize{#1}}
\define@key[psset]{pstricks-add}{ysubticksize}{\edef\psk@ysubticksize{#1}}
%
\define@key[psset]{pstricks-add}{tickwidth}{%
  \pst@getlength{#1}\psk@xtickwidth
  \pst@getlength{#1}\psk@ytickwidth}
\define@key[psset]{pstricks-add}{xtickwidth}{\pst@getlength{#1}\psk@xtickwidth}
\define@key[psset]{pstricks-add}{ytickwidth}{\pst@getlength{#1}\psk@ytickwidth}
\define@key[psset]{pstricks-add}{subtickwidth}{%
  \pst@getlength{#1}\psk@xsubtickwidth
  \pst@getlength{#1}\psk@ysubtickwidth}
\define@key[psset]{pstricks-add}{xsubtickwidth}{\pst@getlength{#1}\psk@xsubtickwidth}
\define@key[psset]{pstricks-add}{ysubtickwidth}{\pst@getlength{#1}\psk@ysubtickwidth}
%
\define@key[psset]{pstricks-add}{tickcolor}{\edef\psk@xtickcolor{#1}\edef\psk@ytickcolor{#1}}
\define@key[psset]{pstricks-add}{xtickcolor}{\edef\psk@xtickcolor{#1}}
\define@key[psset]{pstricks-add}{ytickcolor}{\edef\psk@ytickcolor{#1}}
\define@key[psset]{pstricks-add}{subtickcolor}{\edef\psk@xsubtickcolor{#1}\edef\psk@ysubtickcolor{#1}}
\define@key[psset]{pstricks-add}{xsubtickcolor}{\edef\psk@xsubtickcolor{#1}}
\define@key[psset]{pstricks-add}{ysubtickcolor}{\edef\psk@ysubtickcolor{#1}}
%
\define@key[psset]{pstricks-add}{xticklinestyle}{%
  \@ifundefined{psls@#1}%
    {\@pstrickserr{Line style `#1' not defined}\@eha}%
    {\edef\psxticklinestyle{#1}}%
}
\define@key[psset]{pstricks-add}{xsubticklinestyle}{%
  \@ifundefined{psls@#1}%
    {\@pstrickserr{Line style `#1' not defined}\@eha}%
    {\edef\psxsubticklinestyle{#1}}%
}
\define@key[psset]{pstricks-add}{yticklinestyle}{%
  \@ifundefined{psls@#1}%
    {\@pstrickserr{Line style `#1' not defined}\@eha}%
    {\edef\psyticklinestyle{#1}}%
}
\define@key[psset]{pstricks-add}{ysubticklinestyle}{%
  \@ifundefined{psls@#1}%
    {\@pstrickserr{Line style `#1' not defined}\@eha}%
    {\edef\psysubticklinestyle{#1}}%
}
\define@key[psset]{pstricks-add}{ticklinestyle}{\psset{xticklinestyle=#1,yticklinestyle=#1}}
\define@key[psset]{pstricks-add}{subticklinestyle}{\psset{xsubticklinestyle=#1,ysubticklinestyle=#1}}
\psset{subticksize=0.75,subticks=1,tickcolor=black,ticklinestyle=solid,%
  subticklinestyle=solid,%
  subtickcolor=gray,%
  tickwidth=0.5\pslinewidth,%
  subtickwidth=0.25\pslinewidth}
%
\newif\ifis@yAxis%
%
\def\psxs@@axes#1#2#3#4#5{% llx,lly,urx,ury,exch,arrowA,arrowB
  \begin@SpecialObj%
    \def\pst@tempa{#5}%
    \ifx\pst@tempa\@empty\relax
      \is@yAxisfalse
      \ifnum\psk@dx=\z@
        \pst@dimg=\psk@Dx\psxunit
        \edef\psk@dx{\number\pst@dimg}%
      \fi
    \else
      \is@yAxistrue
      \ifnum\psk@dy=\z@
        \pst@dimg=\psk@Dy\psyunit       
        \edef\psk@dy{\number\pst@dimg}%
      \fi
    \fi% 
    \let\pst@linetype\pst@arrowtype%
    \pst@addarrowdef%
    \addto@pscode{%
      \ifis@yAxis 0 \pst@number#4 \else \pst@number#3 0 \fi
      \ifis@yAxis 0 \pst@number#2 \else \pst@number#1 0 \fi
      ArrowA
      CP 4 2 roll
      ArrowB 
      /yEnd exch def /xEnd exch def
      xEnd yEnd L                                  % the line with arrows 
      /yStart exch def
      /xStart exch def
      stroke                                       % draw the main line
%\psk@ticks: all=0; x=1; y=2; none=3
      \number\psk@ticks\space dup 2 mod 0 eq \ifis@yAxis true \else false \fi and 
      exch 2 lt \ifis@yAxis false \else true \fi and or {
      /viceversa 
        \ifis@yAxis\pst@number#2 \pst@number#4 \else\pst@number#1 \pst@number#3 \fi
         gt { true }{ false } ifelse def           % other way round
      /epsilon 0.01 def                            % rounding errors
      /minTickline \ifis@yAxis \pst@number#1 \else \pst@number#2 \fi def
      /maxTickline \ifis@yAxis \pst@number#3 \else \pst@number#4 \fi def
      /dT \ifis@yAxis\psk@dy\else\psk@dx\fi\space 65536 div viceversa {neg} if def 
      /subTNo \ifis@yAxis\psk@ysubticks\else\psk@xsubticks\fi \space def
      subTNo 0 gt { /dsubT dT subTNo div def}{ /dsubT 0 def } ifelse  % deltaSubTick
      \ifis@yAxis \psk@yticksize \else \psk@xticksize \fi
      /tickend exch def /tickstart exch def
      /Twidth \ifis@yAxis \psk@ytickwidth \else \psk@xtickwidth \fi\space def
      /subTwidth \ifis@yAxis \psk@ysubtickwidth \else \psk@xsubtickwidth \fi\space def
      /STsize \ifis@yAxis \psk@ysubticksize \else \psk@xsubticksize \fi\space def
      /TColor {
        \ifis@yAxis\pst@usecolor\psk@ytickcolor
        \else\pst@usecolor\psk@xtickcolor\fi\space } def
      /subTColor {
        \ifis@yAxis\pst@usecolor\psk@ysubtickcolor
        \else\pst@usecolor\psk@xsubtickcolor\fi\space } def
      /MinValue { \ifis@yAxis yStart \else xStart \fi
        \ifx\psk@arrowA\@empty\else 
          \psk@arrowsize\space CLW mul add \psk@arrowlength\space mul 
           viceversa { sub epsilon add } { add epsilon sub } ifelse \fi } def
      /MaxValue { \ifis@yAxis yEnd \else xEnd \fi
        \ifx\psk@arrowB\@empty\else
          \psk@arrowsize\space CLW mul add \psk@arrowlength\space mul 
           viceversa { add epsilon sub } { sub epsilon add } ifelse \fi } def
      /logLines {
        \ifnum\psk@logLines=\z@ true \else         % all axes
          \ifnum\psk@logLines<\tw@                 % x axis
            \ifis@yAxis false \else true \fi       % do we have x or y axis
          \else
            \ifnum\psk@logLines<\thr@@             % y axis
              \ifis@yAxis true \else false \fi     % do we have x or y axis
            \else 
              false                                % no one
            \fi
          \fi
        \fi
      } def
      /LSstroke {                                  % set linestyle and stroke
        \ifis@yAxis\@nameuse{psls@\psyticklinestyle}
        \else\@nameuse{psls@\psxticklinestyle}\fi } def
      /subLSstroke {                               % set sublinestyle and stroke
        \ifis@yAxis\@nameuse{psls@\psysubticklinestyle}
        \else\@nameuse{psls@\psxsubticklinestyle}\fi } def
%\iffalse
% start ticks ---------------------------------------------------------
      0 dT MaxValue {                              % the positive part of the axes
        /cntTick exch def                          % the index
        logLines {                                 % log lines?
          gsave
          1 10 subTNo div 9.99 {                   % do not write a line for 1
            /dx exch def                           % save index
            /x cntTick dT dx log mul add def       %
            x abs MaxValue abs le {                % out of range?
              x minTickline #5\space moveto
              x maxTickline #5\space lineto
            } if
          } for
          subTwidth SLW subTColor                  % set line width and subtick color
          subLSstroke
          grestore                                 % restore main tick status
          stroke
          /dsubT 0 def                             % no other subticks
        } if
        dsubT abs 0 gt {                           % du we have subticks?
          gsave                                    % save graphic state
% start subticks ----------------------------------------------------------
            /cntsubTick cntTick dsubT add def
            subTNo 1 sub {
            cntsubTick abs MaxValue abs le {       % out of range?
              cntsubTick tickstart STsize mul #5\space moveto
              cntsubTick tickend STsize mul #5\space lineto
            }{ exit }  ifelse
            /cntsubTick cntsubTick dsubT add def
          } repeat 
          subTwidth SLW subTColor                  % set line width and subtick color
          subLSstroke
% end subticks ----------------------------------------------------------
          grestore                                 % restote tick status
        } if
        gsave
        cntTick tickstart #5\space moveto          % line begin main Tick
        cntTick tickend #5\space lineto            % lineto tick end
        Twidth SLW TColor                            % set line width and tick color
        LSstroke
        grestore
      } for
% end ticks ----------------------------------------------------------
%\fi
%\iffalse
% ================================================ % the other side 
      /dT dT neg def                               % the other side of the axis
      /dsubT dsubT neg def
% start ticks ----------------------------------------------------------
      0 dT MinValue epsilon viceversa { add }{ sub } ifelse {
        /cntTick exch def
        logLines {                                 % log lines?
          gsave
          1 10 subTNo div 9.99 {                   % do not write a line for 1
            /dx exch def                           % save index
            /x cntTick dT dx log mul add def       %
            x abs MinValue abs le {                % out of range?
              x minTickline #5\space moveto
              x maxTickline #5\space lineto
            } if
          } for
          /dsubT 0 def 
          subTwidth SLW subTColor                  % set line width and subtick color
          subLSstroke
          grestore
        }
        dsubT abs 0 gt {                           % do we have subticks?
          gsave                                    % save main state
% start subticks ----------------------------------------------------------
          /cntsubTick cntTick dsubT add def
          subTNo 1 sub {
            cntsubTick abs MinValue abs le {       % out of range?
              cntsubTick tickstart STsize mul #5\space moveto
              cntsubTick tickend STsize mul #5\space lineto
            }{ exit } ifelse
            /cntsubTick cntsubTick dsubT add def
          } repeat % for
% end subticks ----------------------------------------------------------
          subTwidth SLW subTColor                  % set line width and subtick color
          subLSstroke
          grestore                                 % restore main state
        } if
        gsave
        cntTick tickstart #5\space moveto         % line begin main Tick
        cntTick tickend #5\space lineto           % lineto tick end
        Twidth SLW TColor                         % set line width and tick color
        LSstroke
        grestore
      } for
% end ticks ----------------------------------------------------------
%\fi    
    } if 
   }%  
  \end@SpecialObj%
}
\def\psxs@frame{%
  \begin@SpecialObj%
    \addto@pscode{%
      0 0 moveto \pst@number\pst@dimc 0 L
      0 \pst@number\pst@dimd 2 copy rlineto L closepath}%
    \pst@stroke%
    \psk@fillstyle%
  \end@SpecialObj%
  \let\psk@arrowA\@empty%
  \let\psk@arrowB\@empty%
%  \pst@dima=0pt\pst@dimb=0pt%
%  \psset{xticksize=0 \pst@dimd,yticksize=0 \pst@dimc}%
  \pst@xticksizeC=\z@\pst@yticksizeC=\z@%
  \psxs@@axes\pst@dima\pst@dimb\pst@dimc\pst@dimd{}%
  \psxs@@axes\pst@dima\pst@dimb\pst@dimc\pst@dimd{exch}%
}
%
% #1:Max/Min in pt from origin; #2:arrow; #3:min; #4:max
\def\pst@hlabels#1#2#3#4{%
  \ifdim#1=\z@\else%                   % start from 0 ?
    \ifx#2\empty\else\advance#1\ifdim#1>\z@-\fi7\pslinewidth\fi%
    \pst@cnta=#1\relax%                % Distance (in sp) to end.
    \divide\pst@cnta\psk@dx\relax%     % Number of ticks/labels
    \ifnum\pst@cnta=\z@\else%
      \pst@dimb=\psk@dx sp%            % Space between ticks.
%        \ifPst@plot@yAxis\else\showorigintrue\fi%
        \ifnum\psk@labels<\tw@ \ifPst@plot@xAxis\pst@@hlabels\fi\fi%
        \showoriginfalse%
    \fi%
  \fi%
}
% Knows \pst@dimb and \pst@cnta
\def\pst@@hlabels{%
  \pst@dimm=\pst@xticksizeC\advance\pst@dimm by \pslabelsep%
  \vbox to\z@{%
    \ifdim\pst@dimd>\z@\vskip\pst@dimm\else\vss\fi%
    \ifnum\pst@cnta<\z@\pst@dimb=-\pst@dimb\fi%
    \hbox to\z@{%
      \ifshoworigin\hbox to \z@{\hss\pst@@@hlabel{\psk@Ox}\hss}\fi%
      \mmultido{\n=\psk@Ox+\psk@Dx}{\pst@cnta}{%
        \hskip\pst@dimb\hbox to \z@{\hss\pst@@@hlabel{\n}\hss}%
      }\hss%
    }%
    \ifdim\pst@dimd>\z@\vss\else\vskip\pst@dimm\fi%
}}%

% #1:Max/Min in pt from origin; #2:arrow; #3:min; #4:max
% \psk@labels 0:all; 1:x; 2:y; 3:none
\def\pst@vlabels#1#2#3#4{%
  \ifdim#1=\z@\else%
    \ifx#2\empty\else\advance#1\ifdim#1>\z@-\fi7\pslinewidth\fi%
    \pst@cnta=#1\relax          %      % Distance (in sp) to end.
    \divide\pst@cnta\psk@dy\relax  %   % Number of ticks/labels
    \ifnum\pst@cnta=\z@\else%
      \pst@dima=\psk@dy sp%            % Space between ticks.
%      \ifPst@plot@xAxis\else\showorigintrue\fi%
      \ifodd\number\psk@labels\else\ifPst@plot@yAxis\pst@@vlabels\fi\fi%
      \showoriginfalse%
    \fi%
  \fi%
}
% \pst@dima: the width between two labels in pt
% \pst@dimc: the coordinate of the origin in pt
% \pst@cnta: the lowest label
\def\pst@@vlabels{%
  \pst@dimm=\pst@yticksizeC%
  \advance\pst@dimm by \pslabelsep%
  \vbox to\z@{%
    \ifnum\pst@cnta>\z@\pst@dima=-\pst@dima\fi%
    \offinterlineskip%
    \ifshoworigin
      \vbox to \z@{\vss\hbox to\z@{%
        \ifdim\pst@dimc>\z@\hss\else\hskip\pst@dimm\fi
        \pst@@@vlabel{\psk@Oy}%
        \ifdim\pst@dimc>\z@\hskip\pst@dimm\else\hss\fi}\vss}%
    \fi%
    \mmultido{\n=\psk@Oy+\psk@Dy}{\pst@cnta}{%
      \vbox to\pst@dima{\vss}
      \vbox to \z@{%
        \vss\hbox to\z@{%
          \ifdim\pst@dimc>\z@%
            \hss\pst@@@vlabel{\n}\hskip\pst@dimm%
          \else%
            \hskip\pst@dimm\pst@@@vlabel{\n}\hss%
          \fi
        }\vss%
    }}%
    \vss%
}}
%
\define@key[psset]{pstricks-add}{nStep}{\edef\psk@nStep{#1}}
\define@key[psset]{pstricks-add}{nStart}{\edef\psk@nStart{#1}}
\define@key[psset]{pstricks-add}{nEnd}{\edef\psk@nEnd{#1}}
\define@key[psset]{pstricks-add}{xStep}{\edef\psk@xStep{#1}}
\define@key[psset]{pstricks-add}{yStep}{\edef\psk@yStep{#1}}
%
\define@key[psset]{pstricks-add}{xStart}{\edef\psk@xStart{#1}}
\define@key[psset]{pstricks-add}{xEnd}{\edef\psk@xEnd{#1}}
\define@key[psset]{pstricks-add}{yStart}{\edef\psk@yStart{#1}}
\define@key[psset]{pstricks-add}{yEnd}{\edef\psk@yEnd{#1}}
%
\define@key[psset]{pstricks-add}{plotNo}{\edef\psk@plotNo{#1}}
\define@key[psset]{pstricks-add}{plotNoMax}{\edef\psk@plotNoMax{#1}}
%
\psset{nStep=1, nStart=0, nEnd={},%
  xStep=0, yStep=0, xStart={}, xEnd={},  yStart={}, yEnd={}, comma=false,%
  plotNo=1,plotNoMax=1}
%
\def\pstScalePoints(#1,#2)#3#4{%
%  xScale | yScale | xOperator | yOperator  
% the operators can be any Postscript code
  \pst@def{ScalePoints}<%
    /y ED /x ED
    counttomark dup dup cvi eq not { exch pop } if
    /m exch def /n m 2 div cvi def
    n {
      #4\space y mul #2\space mul m 1 roll
      #3\space x mul #1\space mul m 1 roll
      /m m 2 sub
      def } repeat>
}
%\pstScalePoints(1,1){}{}% the default -> no special operators
%
\def\listplot@ii#1{%
  \@nameuse{beginplot@\psplotstyle}%
  \addto@pscode{/D {} def mark}%
  #1%
  \addto@pscode{%
    \tx@PreparePoints
    \pst@number\psxunit
    \pst@number\psyunit
    \tx@ScalePoints%
  }%
  \@nameuse{endplot@\psplotstyle}%
}
%
\pst@def{PreparePoints}<{%
  counttomark /m exch def
  /n m \psk@plotNoMax\space 1 add div cvi def
  \psk@plotNoMax\space 1 gt {% multiple data files?
    n {
      \psk@plotNoMax\space \psk@plotNo\space 1 sub neg roll % x yNo y y y ...
      \psk@plotNoMax\space 1 sub { pop } repeat % x yNo
      /m m \psk@plotNoMax\space 1 sub sub def
      m 2 roll
    } repeat
  } if % no multiple data files
%	counttomark /m exch def
%	/n m 2 div cvi def
  /xMax -99999 def /yMax -99999 def
  /xP 0 def /yP 0 def
  m copy
  n {
    /y exch def /x exch def
    xMax x lt { /xMax x def } if
    yMax y lt {/yMax y def } if
    xP x gt { /xP x def } if
    yP y gt { /yP y def } if
  } repeat
%	m 2 roll
  \psk@xStep\space 0 gt \psk@yStep\space 0 gt or (\psk@xStart) length 0 gt or
  (\psk@yStart) length 0 gt or (\psk@xEnd) length 0 gt or (\psk@yEnd) length 0 gt or {
%
    (\psk@xStart) length 0 gt {\psk@xStart\space }{ xP } ifelse /xStart exch def
    (\psk@yStart) length 0 gt {\psk@yStart\space }{ yP } ifelse /yStart exch def
    (\psk@xEnd) length 0 gt { \psk@xEnd\space }{ xMax } ifelse /xEnd exch def
    (\psk@yEnd) length 0 gt { \psk@yEnd\space }{ yMax } ifelse /yEnd exch def
    n {
      m -2 roll
      2 copy /yVal exch def /xVal exch def
      xVal xP ge
      yVal yP ge and
      xVal xEnd le and
      yVal yEnd le and
      xVal xStart ge and
      yVal yStart ge and {
        /xP xP \psk@xStep\space add def
        /yP yP \psk@yStep\space add def
      }{%
        pop pop
        /m m 2 sub def
      } ifelse
    } repeat
  }{%
    /ncount 0 def
    (\psk@nEnd) length 0 gt { \psk@nEnd\space }{ m } ifelse 
    /nEnd exch def
    n {
      m -2 roll
      \psk@nStep\space 1 gt {
        ncount \psk@nStart\space sub \psk@nStep\space mod 0 eq }{ true } ifelse
        ncount nEnd le and
        ncount \psk@nStart\space ge and not {
          pop pop
          /m m 2 sub def
        } if
        /ncount ncount 1 add def
      } repeat
  } ifelse
}>
%
%
\define@key[psset]{pstricks-add}{xAxisLabel}{\edef\psk@xAxisLabel{#1}}
\define@key[psset]{pstricks-add}{yAxisLabel}{\edef\psk@yAxisLabel{#1}}
\psset{xAxisLabel=x,yAxisLabel=y}
\define@key[psset]{pstricks-add}{xAxisLabelPos}{\edef\psk@xAxisLabelPos{#1}}
\define@key[psset]{pstricks-add}{yAxisLabelPos}{\edef\psk@yAxisLabelPos{#1}}
\psset{yAxisLabelPos=\@empty,xAxisLabelPos=\@empty}
%
\newdimen\psk@llx
\newdimen\psk@lly
\newdimen\psk@urx
\newdimen\psk@ury
\define@key[psset]{pstricks-add}{llx}{\pssetxlength\psk@llx{#1}}
\define@key[psset]{pstricks-add}{lly}{\pssetylength\psk@lly{#1}}
\define@key[psset]{pstricks-add}{urx}{\pssetxlength\psk@urx{#1}}
\define@key[psset]{pstricks-add}{ury}{\pssetylength\psk@ury{#1}}
\psset{llx=\z@, lly=\z@, urx=\z@, ury=\z@}% prevents rounding errors 
\newif\ifPst@plot@box
\define@key[psset]{pstricks-add}{box}[true]{\@nameuse{Pst@plot@box#1}}

\newdimen\pst@xunit
\newdimen\pst@yunit
%
%\def\psgraph*{\psgraph{*}}
\def\psgraph{\begingroup\pst@ifstar\psgraph@i}
\def\psgraph@i{%
  \@ifnextchar[{\psgraph@ii}{\psgraph@ii[]}}
\def\psgraph@ii[#1](#2,#3)(#4,#5)#6#7{%
%\def\psgraph{\@ifnextchar[{\psgraph@i}{\psgraph@i[]}}
%\def\psgraph@i[#1](#2,#3)(#4,#5)#6#7{%
% options | minX | minY | maxX | maxY | Length x-axis | length y-axis%  
  \pst@dimo=#4\p@\advance\pst@dimo by -#2\p@ % delta x
  \pst@divide{#6}{\pst@dimo}\pst@tempa
  \pst@xunit=\pst@tempa\p@%
%  
  \pst@dimo=#5\p@\advance\pst@dimo by -#3\p@ % delta y
  \pst@divide{#7}{\pst@dimo}\pst@tempa%
  \pst@yunit=\pst@tempa\p@%
  %
  \pst@dima=#2\pst@xunit \advance\pst@dima by \psk@llx%
  \pst@dimb=#3\pst@yunit \advance\pst@dimb by \psk@lly%
  \pst@dimc=#4\pst@xunit \advance\pst@dimc by \psk@urx%
  \pst@dimd=#5\pst@yunit \advance\pst@dimd by \psk@ury%
  \if@star\pspicture*(\pst@dima,\pst@dimb)(\pst@dimc,\pst@dimd)\else
    \pspicture(\pst@dima,\pst@dimb)(\pst@dimc,\pst@dimd)\fi
  \psset{xunit=\pst@xunit,yunit=\pst@yunit}%
    \psaxes[#1](0,0)(#2,#3)(#4,#5)%
    \ifx\psk@xAxisLabel\@empty\else%
      \ifx\psk@xAxisLabelPos\@empty%
        \uput[0](#4,0){\psk@xAxisLabel}%
        \uput[90](0,#5){\psk@yAxisLabel}%
      \else%
        \rput(\psk@xAxisLabelPos){\psk@xAxisLabel}%
        \rput{90}(\psk@yAxisLabelPos){\psk@yAxisLabel}%
      \fi%
    \fi %
}
\def\endpsgraph{%
  \pst@killglue
  \endpspicture%
  \endgroup%
}
\@namedef{psgraph*}{\psgraph*}
\@namedef{endpsgraph*}{\endpsgraph}

%
\iffalse% still on progress
\def\psIVaxes{\@ifnextchar[{\psIVaxes@i}{\psIVaxes@i[]}}
\def\psIVaxes@i[#1](#2,#3)(#4,#5){%
  \edef\@psxMax{#3}
  \edef\@psyMax{#5}
  \psaxes[#1](#2,#3)(#4,#5)
  \@ifnextchar[{\psIVaxes@ii}{\psIVaxes@ii[]}%
}
\def\psIVaxes@ii[#1](#2,#3){%
  \psset{#1}
  \pst@ticks{90}{#2}{5}{1}
% \pst@ticks{angle}{dx}{n}{int}
% int=1 if ticks appear on top of axes, 0 otherwise.
  \ignorespaces
  \@ifnextchar[{\psIVaxes@iii}{\psIVaxes@iii[]}%
}
\def\psIVaxes@iii[#1](#2,#3){{%  to do ....
}}
\fi
%
%
\newcount\linecnt
\begingroup
\catcode`\,=13
\catcode`\_=13
\gdef\savedata@#1[#2]{%
  \xdef\pst@tempg{#2_}%
  \endgroup
  \let#1\pst@tempg
  \global\let\pst@tempg\relax
  \ignorespaces}
\gdef\readdata@{%
  \read1 to \pst@tempa
  \ifnum\linecnt=\psk@nStep
    \global\linecnt=0
    \expandafter\readdata@@\pst@tempa_\@nil
  \fi
  \global\advance\linecnt by 1
  \ifeof1\else\expandafter\readdata@\fi}
\gdef\pst@@readfile#1#2\@nil{\addto@pscode{,#1#2}}%
\gdef\readdata@@#1#2\@nil{\xdef\pst@tempg{\pst@tempg,#1#2}}%
\endgroup

\def\readdata{\@ifnextchar[{\readdata@i}{\readdata@i[]}}
\def\readdata@i[#1]#2#3{{%
  \def\pst@tempa{#1}%
  \ifx\pst@tempa\@empty\else\psset{#1}\fi
  \openin1=#3
  \begingroup
  \edef\pst@tempg{}%
  \ifeof1
    \@pstrickserr{Data file `#3' not found.}\@ehpa
  \else
    \pst@datadelimiters
    \catcode`\[=1
    \catcode`\]=2
    \global\linecnt=\psk@nStep
    \readdata@%
  \fi
  \endgroup
  \global\let#2\pst@tempg
  \global\let\pst@tempg\relax
  \ignorespaces%
}}
%
% D.G. addition - Jun.  9, 1998 - Polar plots using the \psplot macro
% Code added according the way suggested by Ulrich Dirr <udirr@das-team.com>

% For polar plots
\newif\ifpolarplot
\define@key[psset]{pstricks-add}{polarplot}{\@nameuse{polarplot#1}}
\psset{polarplot=false}
%
\define@key[psset]{pstricks-add}{method}{\edef\psk@method{#1}}%
\define@key[psset]{pstricks-add}{whichabs}{\edef\psk@whichabs{#1}}%
\define@key[psset]{pstricks-add}{whichord}{\edef\psk@whichord{#1}}%
\define@key[psset]{pstricks-add}{plotfuncx}{\edef\psk@plotfuncx{#1}}%
\define@key[psset]{pstricks-add}{plotfuncy}{\edef\psk@plotfuncy{#1}}%
\define@key[psset]{pstricks-add}{expression}{\edef\psk@expression{#1}}%
\define@key[psset]{pstricks-add}{VarStep}{\edef\psk@VarStep{#1}}%
\newif\ifPst@buildvector%
\define@key[psset]{pstricks-add}{buildvector}[true]{\@nameuse{Pst@buildvector#1}}%
\newif\ifPst@algebraic%
\define@key[psset]{pstricks-add}{algebraic}[false]{\@nameuse{Pst@algebraic#1}}%
\psset{method=default, whichabs=default, whichord=default,
  plotfuncx=default, plotfuncy=default, buildvector=false, algebraic=false, VarStep=default}
\def\@rkiv{rk4}%
\def\@adams{adams}%
\psset{dimen=middle}% remark of ML
\def\@undef{undef}%
\def\@default{default}%
%
\def\psplot@i#1#2#3{%
  \pst@killglue
  \begingroup
    \use@par
    \@nameuse{beginplot@\psplotstyle}%
% D.G. modification begin - Jun.  9, 1998
    \ifpolarplot
      \addto@pscode{%
        \psplot@init
        /x #1 def
        /x1 #2 def
        /dx x1 x sub \psk@plotpoints div def
        /F@pstplot \ifPst@algebraic (#3) tx@EqdfDict begin AlgParser end cvx
            \else { #3 } \fi  def 
        /xy {% Adapted from \parametricplot@i
          F@pstplot dup x cos mul exch x sin mul
%          #3 dup x cos mul exch x sin mul
          \pst@number\psxunit mul exch
          \pst@number\psyunit mul exch
        } def}%
    \else
% D.G. modification end
    \addto@pscode{%
      \psplot@init
      /x #1 def
      /x1 #2 def
      /dx x1 x sub \psk@plotpoints div def
      /F@pstplot \ifPst@algebraic (#3) tx@EqdfDict begin AlgParser end cvx
                 \else { #3 } \fi  def 
      /xy {
        x \pst@number\psxunit mul F@pstplot \pst@number\psyunit mul
%       \ifPst@algebraic F@pstplot \else #3 \fi \pst@number\psyunit mul
      } def}%
    \fi
    \gdef\psplot@init{}%
    \@pstfalse
    \@nameuse{testqp@\psplotstyle}%
    \if@pst
      \psplot@ii
    \else
      \psplot@iii
    \fi
  \endgroup
  \ignorespaces}
\def\psplot@ii{%
  \ifx\psk@VarStep\@default
    \addto@pscode{%
      xy \@nameuse{beginqp@\psplotstyle}
      \psk@plotpoints 1 sub {
        /x x dx add def
        xy \@nameuse{doqp@\psplotstyle}
      } repeat
      /x x1 def
      xy \@nameuse{doqp@\psplotstyle}}%
  \else
    \addto@pscode{%
      xy \@nameuse{beginqp@\psplotstyle}
      /Olddt dx def /OldOlddt dx def /Oldy F@pstplot def /y 0 def
      /Ndt dx def /Ny 0 def
      { /Ndt Ndt 1.5 mul def %%new step is 50% larger
        x1 x sub dup Ndt lt { /Ndt exch def } { pop } ifelse 
        { /x x /Ny /x x Ndt add def tx@EqdfDict begin F@pstplot end def def
          /x x /y /x x Ndt 2 div add def tx@EqdfDict begin F@pstplot end def def
              /Oldy tx@EqdfDict begin F@pstplot end def
         Ny y sub Ndt div y Oldy sub Ndt div dup 0 eq { pop pop exit } if div abs 
         dup 1 gt { 1 exch div } if \psk@VarStep\space gt%%if greater than rate OK
         { exit } { /Ndt Ndt 2 div def } ifelse
          Ndt dx lt { exit } if
       } loop
        /x x Ndt add def xy \@nameuse{doqp@\psplotstyle}
        x x1 eq { exit } if
        /Olddt Ndt def
        /OldOlddt Olddt def
       /y Ny def} loop}%
  \fi
  \@nameuse{endqp@\psplotstyle}}
\def\psplot@iii{%
  \ifx\psk@VarStep\@default
    \addto@pscode{%
      mark
      /n 2 def
      \psk@plotpoints {
        xy
        n 2 roll
        /n n 2 add def
        /x x dx add def
      } repeat
      /x x1 def
      xy
      n 2 roll}%
  \else\pst@killglue
    \addto@pscode{%
      /Olddt dx def /OldOlddt dx def /Oldy F@pstplot def /y 0 def
      /Ndt dx def /Ny 0 def /n 2 def
      mark xy
      { /Ndt Ndt 1.5 mul def %%new step is 50% larger
        x1 x sub dup Ndt lt { /Ndt exch def } { pop } ifelse 
        { /x x /Ny /x x Ndt add def tx@EqdfDict begin F@pstplot end def def
          /x x /y /x x Ndt 2 div add def tx@EqdfDict begin F@pstplot end def def
              /Oldy tx@EqdfDict begin F@pstplot end def
         Ny y sub Ndt div y Oldy sub Ndt div dup 0 eq { pop pop exit } if div abs 
         dup 1 gt { 1 exch div } if \psk@VarStep\space gt%%if greater than rate OK
         { exit } { /Ndt Ndt 2 div def } ifelse
          Ndt dx lt { exit } if
       } loop
        /x x Ndt add def xy /n n 2 add def n 2 roll
        x x1 eq { exit } if
        /Olddt Ndt def
        /OldOlddt Olddt def
       /y Ny def} loop}%
  \fi
  \@nameuse{endplot@\psplotstyle}}
\def\parametricplot@i#1#2#3{%
  \pst@killglue%
  \begingroup
    \use@par
    \@nameuse{beginplot@\psplotstyle}%
    \addto@pscode{%
      \psplot@init
      /t #1 def
      /t1 #2 def
      /dt t1 t sub \psk@plotpoints div def
      \ifPst@algebraic/F@pstplot (#3) tx@EqdfDict begin AlgParser end cvx def \fi
      /xy {
        \ifPst@algebraic F@pstplot \else #3 \fi
        \pst@number\psyunit mul exch
        \pst@number\psxunit mul exch
      } def}%
    \gdef\psplot@init{}%
    \@pstfalse
    \@nameuse{testqp@\psplotstyle}%
    \if@pst
      \parametricplot@ii
    \else
      \parametricplot@iii
    \fi
  \endgroup
  \ignorespaces}
%
\define@key[psset]{pstricks-add}{barwidth}{\pst@getlength{#1}\psk@barwidth}
\psset{barwidth=0.25cm}
%
\def\psbar@ii{\addto@pscode{false \tx@NArray \psbar@iii}}
\def\psbar@iii{%
  newpath
  n { 
    /Yval exch def /Xval exch def 
    Xval \number\psk@barwidth 0.5 mul sub 0 moveto 
    0 Yval rlineto \number\psk@barwidth 0 rlineto 
    0 Yval neg rlineto \number\psk@barwidth neg 0 rlineto
  } repeat
}%
\def\beginplot@bar{\begin@SpecialObj}
\def\endplot@bar{%
  \psbar@ii\psk@fillstyle\ifpsshadow\pst@closedshadow\fi%
  \pst@stroke
  \end@SpecialObj%
}
%
\def\resetOptions{%
  \@zero=0%
  \psset{%
%%%% pstricks %%%%%%%%
  unit=1cm,%
  swapaxes=false,%
  showpoints=false,%
  border=0pt, bordercolor=white,%
  doubleline=false, doublesep=1.25\pslinewidth,%
  doublecolor=white,%
  shadow=false, shadowsize=3pt, shadowangle=-45, shadowcolor=darkgray,%
  linewidth=.8pt,%
  linecolor=black,%
  dash=5pt 3pt, dashadjust=true,%
  dotsep=3pt,%
  linestyle=solid,%
  fillcolor=white,%
  hatchwidth=.8pt, hatchsep=4pt, hatchcolor=black, hatchangle=45,%
  fillstyle=none,%
  arrows=-, arrowscale=1, arrowsize=1.5pt 2, arrowlength=1.4, arrowinset=.4,%
  tbarsize=2pt 5,%
  bracketlength=.15, rbracketlength=.15,%
  liftpen=0, linetype=0,%
  gangle=0,%
  curvature=1 .1 0,%
  dotsize=2pt 2,%
  dotangle=0, dotscale=1, dotstyle=*,%
  linearc=0pt,%
  framearc=0,%
  cornersize=relative,%
  dimen=outer,%
  gridwidth=.8pt, griddots=0, gridcolor=black,%
  subgridwidth=.4pt, subgridcolor=gray, subgriddots=0, subgriddiv=5,%
  gridlabels=10pt, gridlabelcolor=black,%
  framesep=3pt, boxsep=true,%
  trimode=U,%
  arcsep=0,radius=.25cm,%
  ref=c,rot=0,labelsep=5pt,refangle=0,%
%%%%%%%%% pst-plot %%%%%%%%%%%%%
  plotstyle=line,plotpoints=50,%
  ticksize=-4pt 4pt,tickstyle=full,ticks=all,%
  labels=all,Ox=0,Dx=1,dx=0,Oy=0,Dy=1,dy=0,%
  showorigin=true,%
  axesstyle=axes,%
  intSeparator={,}%
  braceWidth=0.35,bracePos=0.5,%
  arrowscale=1,%
  ArrowFill=true,%
  ArrowInsidePos=0.5,%
  ArrowInsideNo=1,ArrowInsideOffset=0,%
  dashColorI=black,dashColorII=red,dashNo=0.2,linecap=0,%
  wBegin=\pslinewidth, wEnd=\pslinewidth,%
  lineAngle=0,%
  xyAxes=true,%
  xDecimals={},yDecimals={},%
  xLabel={},yLabel={},%
  xlogBase={},ylogBase={},%
  logLines=none,%
  nStep=1,nStart=0,nEnd={},%
  xStep=0,yStep=0,xStart={},xEnd={},yStart={},yEnd={},comma=false,%
  xAxisLabel=x,yAxisLabel=y,%
  yAxisLabelPos=\@empty,xAxisLabelPos=\@empty,%
  llx=\z@, lly=\z@, urx=\z@, ury=\z@,% prevents rounding errors 
%%%%%%%%% pst-node %%%%%%%%%%
  nodealign=false,%
  href=0,%
  vref=.7ex,%
  framesize=10pt,%
  nodesep=0pt,%
  arm=10pt,%
  offset=0pt,%
  angle=0,%
  arcangle=8,%
  ncurv=.67,%
  loopsize=1cm,%
  boxsize=.4cm,%
  nrot=0,%
  npos=,%
  tpos=0.5,%
  shortput=none,%
  colsep=1.5cm,%
  rowsep=1.5cm,%
%  shortput=tablr,%%
  mcol=c,%
  mnode=R,%
  emnode=none}
}
%
%
\catcode`\@=\PstAtCode\relax
%
%% END: pstricks-add.tex
\endinput
