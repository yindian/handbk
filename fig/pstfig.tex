\input pstricks.tex
\input pst-node.tex
\input pst-plot.tex
\input pst-math.tex
\input multido.tex
\input pstricks-add.tex
\catcode`\@=11
\ifx\LaTeX\@undefined
\else
	\def\nopagenumbers{\pagestyle{empty}}
\fi
\nopagenumbers
\def\psPerpen(#1)(#2)(#3)#4{%	#4 <- point(#1) perpendicular foot to line(#2, #3)
%	      o   (a, b)  #2
%	      |
%	   k  |(x,y)
%	o-----+-----o
%	(x1,y1)    (x2,y2)
%	k = ( (a-x1)(x2-x1) + (b-y1)(y2-y1)) / ( (x2-x1)^2 + (y2-y1)^2 )
%	x = k(x2 - x1) + x1, y = k(y2 - y1) + y1
  \pst@getcoor{#1}\pst@tempa%
  \pst@getcoor{#2}\pst@tempb%
  \pst@getcoor{#3}\pst@tempc%
  \pnode(!%
    \pst@tempa /YA exch \pst@number\psyunit div def
    /XA exch \pst@number\psxunit div def
    \pst@tempb /YB exch \pst@number\psyunit div def
    /XB exch \pst@number\psxunit div def
    \pst@tempc /YC exch \pst@number\psyunit div def
    /XC exch \pst@number\psxunit div def
    /getk
    XA XB sub XC XB sub mul
    YA YB sub YC YB sub mul
    add
    XC XB sub dup mul YC YB sub dup mul add
    div def
    XC XB sub getk mul XB add
    YC YB sub getk mul YB add){#4}
}%psPerpen
\def\markang{\pst@object{markang}}% \markang{arrow}(A)(O)(B){radius}
%          / B
%        /
%      /  )
%   O ----+---- A
\def\markang@i#1(#2)(#3)(#4)#5{%
  \begin@SpecialObj
  \setLCNode(#2){1}(#3){-1}{@temp@OA}
  \setLCNode(#4){1}(#3){-1}{@temp@OB}
  \def\@temp@arrow{#1}
  \ifx\@temp@arrow\@empty
    \def\@temp@arrow{-}
  \fi
  \psarc{\@temp@arrow}(#3){#5}{(@temp@OA)}{(@temp@OB)}
  \end@SpecialObj
} %
\def\markangrt{\pst@object{markangrt}}% \markangrt(A)(O)(B){radius}
%     | B
%     |   c
%    b|--+
%     |  |a
%   O +--+----- A
\def\markangrt@i(#1)(#2)(#3)#4{%
  \begin@SpecialObj
  \setLDNode(#2)(#1){#4}{@temp@a}
  \setLDNode(#2)(#3){#4}{@temp@b}
  \setLCNode(@temp@a){1}(#2){-1}{@temp@Oa}
  \setLCNode(@temp@b){1}(#2){-1}{@temp@Ob}
  \setLCNode(@temp@Oa){1}(@temp@Ob){1}{@temp@Oc}
  \setLCNode(#2){1}(@temp@Oc){1}{@temp@c}
  \psline(@temp@a)(@temp@c)(@temp@b)
  %\pcangle[angleA=(@temp@OB),angleB=(@temp@OA),arm=0pt]{\@temp@arrow}(@temp@a)(@temp@b)
  \end@SpecialObj
} %
%   #1-------#4----------------#2
% where #1#4= #3
%
\def\setLDNode(#1)(#2)#3#4{%
  \pst@getcoor{#1}\pst@tempa%
  \pst@getcoor{#2}\pst@tempb%
  \pssetlength\pst@dima{#3}%
  \pnode(!%
    \pst@tempa /YA exch \pst@number\psyunit div def
    /XA exch \pst@number\psxunit div def
    \pst@tempb /YB exch \pst@number\psyunit div def
    /XB exch \pst@number\psxunit div def
    /dx XB XA sub def
    /dy YB YA sub def
    /dR dx dup mul dy dup mul add sqrt def
    /Length \pst@number\pst@dima \pst@number\psunit div def
    XA dx dR div Length mul add YA dy dR div Length mul add){#4}
}
\def\nLput{\pst@object{nLput}}
\def\nLput@i(#1)(#2)#3#4{%
  \begin@SpecialObj
  \setLNode(#1)(#2){#3}{@temp@lnput}
  \pcline[linestyle=none](#1)(@temp@lnput)%
  \ncput[npos=1]{#4}%
  \end@SpecialObj
} %
\def\nlput{\pst@object{nlput}}
\def\nlput@i(#1)(#2)#3#4{%
  \begin@SpecialObj
  \setLDNode(#1)(#2){#3}{@temp@lnput}
  \pcline[linestyle=none](#1)(@temp@lnput)%
  \ncput[npos=1]{#4}%
  \end@SpecialObj
} %
\def\drawaxes{\@ifnextchar[{\@drawaxes}{\@drawaxes[]}}
\def\@drawaxes[#1]#2#3#4#5{%
\psaxes[linewidth=0.4pt,labels=none,ticks=none,arrowlength=5,arrowinset=0.2,arrowsize=1.8pt,#1]{->}(0,0)(#2,#4)(#3,#5)
\uput{0.2ex}[0](#3,0){x}
\uput{0.8ex}[-30](0,#5){y}
}
\def\markxlabel#1#2#3{%	\markxlabel{pos}{label}{offset}
\uput[-90](#1, 0){\hskip #3 #2}
}
\def\markylabel#1#2#3{%	\markxlabel{pos}{label}{offset}
\uput{1pt}[0](0, #1){\hskip #3 #2}
}
\newdimen\scu@tmpdimd
\def\setcoorunit#1#2{%	\setcoorunit{visual width}{actual width}
\scu@tmpdimd=1pt
\pst@divide{#1\scu@tmpdimd}{#2\scu@tmpdimd}\tmp
\scu@tmpdimd=\tmp 2em
\psset{unit=\scu@tmpdimd}
}
\def\pinum{3.14159265359}
\SpecialCoor
\psset{linewidth=0.8pt,labelsep=0.5ex,unit=2em,algebraic=true,plotstyle=curve}
\def\thickline{\psset{linewidth=0.8pt}}
\def\thinline{\psset{linewidth=0.4pt}}
\def\solidline{\psset{linestyle=solid}}
\def\dashline{\psset{linestyle=dashed}}
\def\noneline{\psset{linestyle=none}}
\def\/#1/{\edef\@mmmm{\noexpand#1}\@mmmm}
\def\frac#1#2{{\begingroup #1\endgroup \over #2}}
\def\dfrac#1#2{{\displaystyle\frac{#1}{#2}}}
\def\nfrac#1#2{{\displaystyle\frac{\raise-0.5ex\hbox{$#1$}}{\raise0.2ex\hbox{$#2$}}}}
\def\nfrac@t#1#2#3{\frac{\raise-0.5ex\hbox{$\@nameuse{#3}#1$}}{\raise0.2ex\hbox{$\@nameuse{#3}#2$}}}
\let\oldnfrac\nfrac
\def\nfrac#1#2{\mathchoice{\oldnfrac#1#2}{\oldnfrac#1#2}{\nfrac@t#1#2{scriptstyle}}{\nfrac@t#1#2{scriptscriptstyle}}}
\endinput
